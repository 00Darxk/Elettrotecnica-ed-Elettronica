\documentclass{article}

\usepackage{cancel}
\usepackage{tikz}
\usepackage{amsmath}
\usepackage{geometry}
\usepackage{graphicx}
\usepackage{amsfonts} 
\usepackage{verbatim}
\usepackage{mathrsfs}  
\usepackage{lmodern}
\usepackage{braket}
\usepackage{bookmark}
\usepackage{circuitikz}
\usepackage[italian]{babel}


%% segnati con un doppio segno percentuale ("%%") le correzioni o inserimenti da apportare al testo

\hypersetup{
    colorlinks=true,
    linkcolor=black,
}

\renewcommand{\contentsname}{Indice}

\tikzset{block/.style = {draw, fill=white, very thick, rectangle, minimum height=1cm, minimum width=2cm},
         lblock/.style={draw,fill=white,very thick, rectangle, minimum height=3cm, minimum width=1cm},
         sum/.style= {draw, fill=white, very thick, circle, node distance=0.5cm}}


\title{Esercizi Svolti su Circuiti a Corrente Continua}
\author{Giacomo Sturm}
\date{AA: 2023/2024 - Ing. Informatica}


\begin{document}

\maketitle

\vspace{10mm}

\begin{center}
    Sorgente del file LaTeX disponibile su \url{https://github.com/00Darxk/Elettrotecnica-ed-Elettronica}
\end{center}

\clearpage

\tableofcontents

\clearpage

\section{Esercitazione del 27 Novembre: Circuiti del Primo Ordine}

\subsection{Esercizio 6.3}
Determinare l'espressione della tensione a vuoto $v_0(t)$ del seguente circuito:
\begin{center}
    \begin{circuitikz}
        \draw (6,0)node[right]{$-$} to[short,*-] (0,0)
                    to[european voltage source=$E$](0,2)
                    to[closing switch](2,2)
                    to[R=$R_1$](4,2)
                    to[R=$R_2$,-*](6,2)node[right]{$+$};
        \draw (4,2) to[R=$R_3$](4,0);
        \draw (2,2)node[above left]{$A$} to[short,*-] (2,3)
                    to[C=$C$](5.8,3)node[above]{$v_C(0^-)=4\,V$}
                    to[short,-*](5.8,2)node[above right]{$B$};
        \draw[->] (6,0.25)--(6,1.75)node[midway, right]{$v_0(t)$};
    \end{circuitikz}
\end{center}
\begin{gather*}
    E=10\,V\\
    R_1=R_2=2\,\Omega\\
    R_3=1\,\Omega\\
    C=1\,F
\end{gather*}

Per risolvere il circuito bisogna isolare l'elemento con memoria ed applciare il teorema di Thevenin o Northon sul circuito, considerando l'elemento con memoria esterno al 
circuito. Dopo aver calcolato la tensione ai capi del condensatore, considerandolo in entrata ad un circuito nei morsetti $A$ e $B$, si può sostituire l'elemento con un 
generatore di tensione equivalente. Per calcolare la tensione a vuoto tra i morsetti $A$ e $B$ si può riscrivere il circuito come:

\begin{center}
    \begin{circuitikz}
        \draw (2,0) to[short](0,0) 
                    to[european voltage source=$E$](0,2)
                    to[R=$R_1$](2,2)
                    to[R=$R_2$](4,2);
        \draw (2,2) to[R=$R_3$](2,0);
        \draw (0,2) to[short,-*](0,2.5)node[above]{$A$};
        \draw (4,2) to[short,-*](4,2.5)node[above]{$B$};
    \end{circuitikz}
\end{center}

Si calcola la tensione a vuoto tramite la formula per i partitori di tensioni:
\begin{gather*}
    E_{th}=\displaystyle\frac{R_1}{R_1+R_3}E=\frac{20}{3}\,V
\end{gather*}
La resistenza $R_2$ non influisce poiché è connessa a vuoto, per cui non può agire sulla tensione, non essendo attraversata da una corrente. 

Si rende la rete passiva per calcolare la resistenza equivalente: 
\begin{center}
    \begin{circuitikz}
        \draw (2,0) to[short](0,0) 
                    to[short](0,2)
                    to[R=$R_1$](2,2)
                    to[R=$R_2$](4,2);
        \draw (2,2) to[R=$R_3$](2,0);
        \draw (0,2) to[short,-*](0,2.5)node[above]{$A$};
        \draw (4,2) to[short,-*](4,2.5)node[above]{$B$};
    \end{circuitikz}
\end{center}
La resistenza equivalente risulta quindi: 
\begin{gather*}
    R_{eq}=\displaystyle\frac{R_1R_3}{R_1+R_3}+R_2=\frac{8}{3}\,\Omega
\end{gather*}

Si può esprimere la tensione ai capi del condensatore mediante la formula precedentemente dimostrata, oppure si può dimostrare la formula 
in questo specifico caso. Si ottiene quindi:
\begin{gather*}
    v_C(t)=\left[v_C(0^-)-E_{th}\right]e^{-\frac{t}{R_{eq}C}}+E_{th}\\
    v_C(t)=\left[\displaystyle 4-\frac{20}{3}\right]e^{-\frac{3}{8}t}+\frac{20}{3}\,V\\
    v_C(t)=-\displaystyle\frac{8}{3}e^{-\frac{3}{8}t}+\frac{20}{3}\,V
\end{gather*}

Dopo aver sostituito il condensatore con un generatore di tensione equivalente si considera una maglia dove sia presente la tensione incognita, per applicare il 
secondo principio di Kirchhoff:
\begin{gather*}
    E-v_C(t)=v_0(t)\\
    10-v_C(t)=v_0(t)\\
    10+\displaystyle\frac{8}{3}e^{-\frac{3}{8}t}-\frac{20}{3}\,V=v_0(t)
\end{gather*}
La tensione a vuoto è quindi:
\begin{gather}
    v_0(t)=\displaystyle\frac{8}{3}e^{-\frac{3}{8}t}+\frac{10}{3}\,V
\end{gather}

\subsection{Esercizio 6.4}
Determinare l'espressione della tensione a vuoto del seguente circuito, a regime prima della chiusura dell'interruttore:
\begin{center}
    \begin{circuitikz}
        \draw (4.2,0) to[short,*-](0,0)
                    to[european voltage source=$E$](0,4)
                    to[R=$R_1$](2,4)
                    to[R=$R_2$](2,2)
                    to[closing switch](2,0);
        \draw (2,4) to[R=$R_3$](4,4)
                    to[C=$C$](4,0);
        \draw (4,4) to[short,-*](4.2,4);
    \end{circuitikz}
\end{center}
\begin{gather*}
    R_1=3\,\Omega\\
    R_2=R_3=6\,\Omega\\
    E=3\,V\\
    C=0.05\,F
\end{gather*}

A regime permanente la tensione del condensatore corrisponde alla tensione erogata dal generatore, per cui:
\begin{gather*}
    v_C(0^-)=3\,V
\end{gather*}
Si applica il teorema di Thevenin senza considerare il condensatore. Si ottiene una tensione equivalente:
\begin{gather*}
    E_{th}=\displaystyle\frac{R_2}{R_1+R_2}E=2\,V
\end{gather*}
La resistenza equivalente corrisponde a:
\begin{gather*}
    R_{eq}=\displaystyle\frac{R_1R_2}{R_1+R_2}+R_3=8\,\Omega
\end{gather*}
Per cui la tensione ai capi del condesnatore corrisponde a:
\begin{gather}
    v_C(t)=[v_C(0^-)]e^{-\frac{t}{R_{eq}C}}+E_{th}=2+e^{-2.5\,t}\,V
\end{gather}

\subsection{Esercizio 6.5}
Determinare la tensione a vuoto del seguente cirucito:
\begin{center}
    \begin{circuitikz}
        \draw (8,0)node[right]{$-$} to[short,*-](0,0)
                    to[european current source=$I$](0,2)
                    to[short](2,2)
                    to[closing switch](2,0);
        \draw (2,2) to[short](4,2)
                    to[R=$R$](4,0);
        \draw (4,2)node[above left]{$A$} to[R=$R$,*-*](6,2)node[above right]{$B$}
                    to[R=$R$](6,0);
        \draw (6,2) to[short,-*](8,2)node[right]{$+$};
        \draw (4,2) to[short](4,4)
                    to[L=$L$](6,4)
                    to[short](6,2);
        \draw (5,0)node[above]{$0$}--(5,-0.5)node[ground]{};
    \end{circuitikz}
\end{center}
\begin{gather*}
    I=2\,A\\
    R=2\,\Omega\\
    L=2\,H
\end{gather*}
A regime l'induttore contiene tutta la carica erogata dal generatore:
\begin{gather*}
    i_L(0^-)=2\,A
\end{gather*}
Si risovle mediante il metodo dei nodi:
\begin{gather*}
    \begin{bmatrix}
        2&-1\\-1&2
    \end{bmatrix}\begin{bmatrix}
        V_{A0}\\V_{B0}
    \end{bmatrix}
    =R\begin{bmatrix}
        2\\0
    \end{bmatrix}\\
    \begin{bmatrix}
        V_{A0}\\V_{B0}
    \end{bmatrix}=\displaystyle\frac{1}{3}
    \begin{bmatrix}
        2&1\\1&2
    \end{bmatrix}\begin{bmatrix}
        4\\0
    \end{bmatrix}=\frac{1}{3}\begin{bmatrix}
        8\\4
    \end{bmatrix}\,V
\end{gather*}
Per cui la tensione equivalente della rappresentazione di Thevenin si ottiene come:
\begin{gather*}
    E_{th}=V_{A0}-V_{B0}=\displaystyle\frac{4}{3}\,V
\end{gather*}
Si misura la resistenza equivalente rendedno passiva la rete, si ottiene quindi:
\begin{gather*}
    R_{eq}=\displaystyle\frac{(R+R)R}{(R+R)+R}=\frac{4}{3}\,\Omega
\end{gather*}

La corrente di cortocircuito di una forma Thevenin collegata a vuoto corrispone alla corrente del generatore di una forma Northon equivalente:
\begin{gather*}
    I_{no}=\displaystyle\frac{E_{th}}{R_{eq}}=1\,A
\end{gather*}
Si determina la tensione ai capi dell'induttore alla chiusura dell'induttore: 
\begin{gather*}
    i_L(t)=\left[\displaystyle i_L(0^-)-\frac{E_{th}}{R_{eq}}\right]e^{-\frac{R_{eq}}{L}t}+\frac{E_{th}}{R_{eq}}=1e^{-\frac{2}{3}t}+1\,A
\end{gather*}
Si sostituisce ora l'induttore con un generatore di corrente che eroga la corrente equivalente ai capi dell'induttore. Si applica quindi di nuovo il metodo dei nodi in questa 
nuova situazione:
\begin{gather*}
    \begin{bmatrix}
        2&-1\\-1&2
    \end{bmatrix}\begin{bmatrix}
        V_{A}\\V_{B}=V_{out}
    \end{bmatrix}=
    R\begin{bmatrix}
        2-i_L\\i_L
    \end{bmatrix}\\
    \begin{bmatrix}
        V_{A}\\V_{out}
    \end{bmatrix}=\displaystyle\frac{1}{3}
    \begin{bmatrix}
        2&1\\1&2
    \end{bmatrix}\begin{bmatrix}
        4-2i_L\\i_L
    \end{bmatrix}=\begin{bmatrix}
        8-4i_L+2i_L\\
        4-2i_L+4i_L
    \end{bmatrix}
\end{gather*}
La tensione ai capi del circuito è quindi: 
\begin{gather}
    V_{out}=\displaystyle\frac{1}{3}\left(4+2i_L\right)=\frac{4}{3}+\frac{2}{3}e^{-\frac{2}{3}t}+\frac{2}{3}=2+\frac{2}{3}e^{-\frac{2}{3}t}\,V
\end{gather}

\subsection{Esercizio 6.6}
Calcolare la corrente $i_0(t)$ considerando la memoria nulla per il condensatore:
\begin{center}
    \begin{circuitikz}
        \draw (8,2) to[short,i=$i_0(t)$](8,0)
                    to[short](0,0)
                    to[european voltage source=$E$](0,2)
                    to[closing switch](2,2)
                    to[R=$R$](4,2)
                    to[C=$C$](4,0);
        \draw (4,2) to[short] (6,2)
                    to[R=$R$](6,0);
        \draw (6,2) to[R=$R$](8,2);    
    \end{circuitikz}
\end{center}
\begin{gather*}
    R=1\,\Omega\\
    C=1\,F\\
    E=1\,V
\end{gather*}

Per le formule sui partitori di tensione si ottiene una tensione equivalente:
\begin{gather*}
    E_{th}=\displaystyle\frac{R\frac{RR}{R+R}}{R+\frac{RR}{R+R}}E=\frac{\frac{1}{2}}{1+\frac{1}{2}}\cdot1\,V=\frac{1}{3}\,V
\end{gather*}
La resistenza equivalente corrisponde a:
\begin{gather*}
    R_{eq}=\displaystyle\frac{R\frac{RR}{R+R}}{R+\frac{RR}{R+R}}\frac{1}{3}\,\Omega
\end{gather*}
Il tempo caratteristico corrisponde a:
\begin{gather*}
    \tau=R_{eq}C=\displaystyle\frac{1}{3}\,s
\end{gather*}
La tensione ai capi del condensatore si ottiene come:
\begin{gather*}
    v_C(t)=(v_C(0^-)-E_{th})e^{-\frac{t}{\tau}}-E_{th}=-\displaystyle\frac{1}{3}e^{-3t}+\frac{1}{3}\,V
\end{gather*}
Per cui l'espressione della corrente risulta essere il rapporto tra la tensione ed il valore della resistenza del resistore ai cui capi è presente questa tensione. 
La corrente non cambia per il resistore in parallelo al lato dove fluisce $i_0$. 
\begin{gather}
    i_0(t)=\displaystyle\frac{v_c(t)}{R}=\frac{1}{3}\left(1-e^{-3t}\right)
\end{gather}

\subsection{Esercizio 6.2}
Calcolare la tensione a vuoto $v_0(t)$ della seguente rete:
\begin{center}
    \begin{circuitikz}
        \draw (8,0)node[right]{$-$} to[short,*-](0,0)
                    to[european voltage source=$E$](0,2)
                    to[closing switch](2,2)
                    to[R=$R_1$](4,2)
                    to[C=$C$](4,0)node[below]{$v_C(0^-)=0\,V$};
        \draw (4,2) to[short] (6,2)
                    to[R=$R_2$](6,0);
        \draw (6,2) to[short,-*](8,2)node[right]{$+$};   
        \draw[->] (8,0.25)--(8,1.75)node[midway, right]{$v_0(t)$};
    \end{circuitikz}
\end{center}
\begin{gather*}
    E=10\,V\\
    R_1=R_2=1\,\Omega\\
    C=1\,F
\end{gather*}
Si considera la rappresentazione equivalente di Thevenin, per individuare la tensione ai capi del condensatore:
\begin{gather*}
    E_{th}=\displaystyle\frac{R_2}{R_1+R_2}E=5\,V
\end{gather*}
Mentre si ottiene una resistenza equivalente data dalle due resistenze in parallelo:
\begin{gather*}
    R_{eq}=\displaystyle\frac{R_1R_2}{R_1+R_2}=\frac{1}{2}\,\Omega
\end{gather*}
La tensione ai capi del condensatore è quindi:
\begin{gather*}
    v_C(t)=(v_C(0^-)-E_{th})e^{-\frac{t}{R_{eq}C}}+E_{th}=\displaystyle -5e^{-2t}+5\,V
\end{gather*}
Il condensatore è montato in parallelo ai morsetti in entrata al circuito, per cui l'espressione della tensione a vuoto coincide con l'espressione del condensatore:
\begin{gather*}
    v_0(t)=5(1-e^{-2t})\,V
\end{gather*}

\subsection{Esercizio 6.1}
Calcolare la costante di tempo $\tau$ dei seguenti circuiti:
\begin{center}
    \begin{circuitikz}
        \draw (2,2) to[short](2,1)
                    to[R=$R_2$](4,1)
                    to[short](4,0)
                    to[short](0,0)
                    to[european voltage source=$E$](0,2)
                    to[R=$R_1$](2,2)
                    to[short](2,3)
                    to[C=$C$](4,3)
                    to[short](4,1);
    \end{circuitikz}
\end{center}
Per trovare il tempo caratteristico è necessario solamente determinare la resistenza equivalente, e l'eccitazione del circuito. Escludendo il condensatore dalla rete 
resa passiva si trova una resistenza equivalente corrispondente alle due resistenze $R_1$ e $R_2$ in parallelo:
\begin{gather*}
    R_{eq}=\displaystyle\frac{R_1R_2}{R_1+R_2}
\end{gather*}
Per cui il tempo caratteristico $\tau$ è:
\begin{gather}
    \tau=CR_{eq}=\displaystyle\frac{CR_1R_2}{R_1+R_2}
\end{gather}

\begin{center}
    \begin{circuitikz}
        \draw (2,2) to[short](4,2)
                to[C=$C$](4,0)
                to[short](0,0)
                to[european current source=$I$](0,4)
                to[short](2,4)
                to[R=$R_1$](2,2)
                to[R=$R_2$](2,0);
    \end{circuitikz}
\end{center}
Rendendo passiva la rete si osserva che la resistenza $R_1$ è collegata al vuoto, per cui la resistenza equivalente coincide con la resistenza $R_2$, per cui il 
tempo caratteristico corrisponde a:
\begin{gather}
    \tau=CR_2
\end{gather}

\begin{center}
    \begin{circuitikz}
        \draw (0,3) to[R=$R$](2,3)
                    to[R=$R$](4,3)
                    to[short](4,1)
                    to[R=$R$](2,1)
                    to[R=$R$](0,1)
                    to[short](0,0)
                    to[european voltage source=$E$](8,0)
                    to[short](8,3)
                    to[R=$R$](6,3)
                    to[short](6,1)
                    to[L=$L$](8,1);
        \draw (6,2) to[short](4,2);
        \draw (2,3) to[R=$R$](2,1);
        \draw (0,1) to[short](0,3);
    \end{circuitikz}
\end{center}
Si considera la rete resa passiva, escludendo l'elemento con memoria:
\begin{center}
    \begin{circuitikz}
        \draw (0,0) to[R,-*](2,0)
                    to[R](4,0)
                    to[short](4,2)
                    to[R](2,2)
                    to[R,*-*](0,2)node[left]{$A$}
                    to[short](0,3)
                    to[short](6,3)
                    to[short](6,2)node[above right]{$A$}
                    to[R,*-*](4,2)node[above]{$C$};
        \draw (2,0)node[below]{$B$} to[R](2,2)node[above]{$D$};
        \draw (6,2) to[short](6.5,2)
                    to[short](6.5,1.5)node[ground]{};
        \draw (0,0) to[short](0,2);
    \end{circuitikz}
\end{center}
Si considera il nodo $A$ il nodo di salto. Si considera una corrente di $1\,A$ inserita tra i nodi $C$ e $A$, e si risolve mediante il metodo dei nodi:
\begin{gather*}
    \begin{bmatrix}
        3&-1&-1\\-1&3&-1\\-1&-1&3
    \end{bmatrix}\begin{bmatrix}
        V_B\\V_C\\V_D
    \end{bmatrix}=R\begin{bmatrix}
        0\\1\\0
    \end{bmatrix}\\
    \begin{bmatrix}
        V_B\\V_C\\V_D
    \end{bmatrix}=\begin{bmatrix}
        0.5&0.25&0.25\\0.25&0.5&0.25\\0.25&0.25&0.5
    \end{bmatrix}R\begin{bmatrix}
        0\\1\\0 
    \end{bmatrix}\\
    V_C=\frac{R}{2}
\end{gather*}
Per cui la resistenza equivalente è:
\begin{gather*}
    R_{eq}=\displaystyle\frac{V_C}{1}=\frac{R}{2}
\end{gather*}
Allora il tempo caratteristico di questo circuito corrisponde a:
\begin{gather}
    \tau=\displaystyle\frac{L}{R_{eq}}=\frac{2L}{R}
\end{gather}

\begin{center}
    \begin{circuitikz}
        \draw (0,3) to[R=$R$](2,3)
                    to[R=$R$](4,3)
                    to[short](4,1)
                    to[R=$R$](2,1)
                    to[R=$R$](0,1)
                    to[short](0,0)
                    to[european current source=$I$](8,0)
                    to[short](8,3)
                    to[R=$R$](6,3)
                    to[short](6,1)
                    to[L=$L$](8,1);
        \draw (6,2) to[short](4,2);
        \draw (2,3) to[R=$R$](2,1);
        \draw (0,1) to[short](0,3);
    \end{circuitikz}
\end{center}
Poiché in questo caso rendendo la rete passiva si sostituisce al generatore di corrente un cortocircuito, il lato formato da sole resistenze connesso in serie al generatore 
di corrente sono connesse a vuoto. Per cui non forniscono contributi, per cui l'unica resistenza utile è quella montata in parallelo all'induttore, per cui la 
resistenza equivalente coincide a questa resistenza:
\begin{gather*}
    R_{eq}=R
\end{gather*}
Per cui il tempo caratteristico corrisponde a:
\begin{equation}
    \tau=\displaystyle\frac{R_{eq}}{L}=\frac{R}{L}
\end{equation}



\end{document}