\documentclass{article}

\usepackage{cancel}
\usepackage{tikz}
\usepackage{amsmath}
\usepackage{geometry}
\usepackage{graphicx}
\usepackage{amsfonts} 
\usepackage{verbatim}
\usepackage{mathrsfs}  
\usepackage{lmodern}
\usepackage{braket}
\usepackage{bookmark}
\usepackage{circuitikz}
\usepackage[italian]{babel}


%% segnati con un doppio segno percentuale ("%%") le correzioni o inserimenti da apportare al testo

\hypersetup{
    colorlinks=true,
    linkcolor=black,
}

\renewcommand{\contentsname}{Indice}

\tikzset{block/.style = {draw, fill=white, very thick, rectangle, minimum height=1cm, minimum width=2cm},
         lblock/.style={draw,fill=white,very thick, rectangle, minimum height=3cm, minimum width=1cm},
         sum/.style= {draw, fill=white, very thick, circle, node distance=0.5cm}}


\title{Esercizi Svolti su Circuiti a Corrente Alternata}
\author{Giacomo Sturm}
\date{AA: 2023/2024 - Ing. Informatica}


\begin{document}

\maketitle

\vspace{10mm}

\begin{center}
    Sorgente del file LaTeX disponibile su \url{https://github.com/00Darxk/Elettrotecnica-ed-Elettronica}
\end{center}

\clearpage

\tableofcontents

\clearpage

\section{Esercizi Svolti il 24 Novembre}

\subsection{Esercizio 1}
Calcolare i valori di resistenza e capacità equivalente, del seguente circuito in regime sinusoidale alla pulsazione di $\omega=2\,\mbox{rad}/s$:
\begin{center}
    \begin{circuitikz}
        \draw (0,2)node[above]{$A$} to[R=$R_1$,*-] (2,2)  
                    to[L=$L_1$](4,2)
                    to[R=$R_2$](4,0)
                    to[short,-*](0,0)node[below]{$B$};
        \draw (2,2) to[C=$C_2$](2,0);
    \end{circuitikz}
\end{center}
\begin{gather*}
    C_2=0.5\,F\\
    L_1=1\,H\\
    R_1=R_2=1\,\Omega
\end{gather*}
Si può risolvere applicando trasformazioni in serie ed in parallelo dell'impedenza:
\begin{gather*}
    z_1=j\omega L_1+R_2=2j+1\\
    z_2=-j\displaystyle\frac{1}{\omega C_2}=-j\\
    z_3=R_1=1
\end{gather*}
Si considera il parallelo tra $z_1$ e $z_2$:
\begin{gather*}
    z_{p12}=\displaystyle\frac{z_1z_2}{z_1+z_2}=\frac{-j(1+2j)}{1+2j-j}=\frac{-j+2}{1+j}\cdot\frac{1-j}{1-j}\\
    z_{p12}=\frac{1-3j}{2}
\end{gather*}
Si considera la serie tra $z_{p12}$ e $z_3$:
\begin{gather*}
    z_{AB}=z_{p12}+z_3=\displaystyle\frac{1}{2}+j\frac{3}{2}+1=\frac{3}{2}+j\frac{3}{2}
\end{gather*}
\begin{gather}
    R_{eq}=\displaystyle\frac{3}{2}\,\Omega\\
    C_{eq}=\frac{2}{3}\,\Omega\cdot \omega=\frac{1}{3} F
\end{gather}

Alternativamente per trovare l'impedenza equivalente della rappresentazione Thevenin si considera un generatore che ergoa una tensione $\overline{V}_{in}$, e si risolve 
mediante il metodo dei nodi o delle maglie:
\begin{center}
    \begin{circuitikz}
        \draw (0,2) to[R=$R_1$,*-] (2,2)node[above]{$A$} 
                    to[L=$L_1$](4,2)
                    to[R=$R_2$](4,0)
                    to[short,-*](0,0);
        \draw (2,2) to[C=$C_2$](2,0);
        \draw (0,0) to[short](-2,0)
                    to[sV=$\overline{V}_{in}$](-2,2)
                    to[short,i=$\overline{I}_{in}$](0,2);
        \draw (3,0)node[above]{$0$} to[short](3,-0.5)node[ground]{};
    \end{circuitikz}
\end{center}

La matrice delle ammettenze modali diventa solamente l'autoammettenza del nodo $A$:
\begin{gather*}
    (y_1+y_2+y_3)\overline{V}_A=\overline{V}_{in}y_3\\
    \overline{I}_{in}=\displaystyle\frac{\overline{V}_{in}-\overline{V}_A}{z_3}\\
    z_{in}=\displaystyle\frac{\overline{V}_{in}}{\overline{I}_{in}}
\end{gather*}
Si calcola ora numericamente:
\begin{gather*}
    \overline{V}_{in}=1\,V\\
    \overline{V}_A=\displaystyle\frac{y_3}{y_1+y_2+y_3}\cdot 1\,V\\
    \overline{V}_A=\displaystyle\frac{1}{\frac{1}{1+2j}-\frac{1}{j}+1}=\displaystyle\frac{1}{\frac{j-1-2j-j(1+2j)}{j(1+2j)}}=\frac{j(1+2j)}{j-1-2j+j-2}\\
    \overline{V}_A=\frac{j-2}{-3}\\
    \overline{I}_{in}=1+\displaystyle\frac{j-2}{3}=\frac{3+j-2}{3}=\frac{1+j}{3}\\
    z_{in}=\displaystyle\frac{1}{1+j}\cdot\frac{1-j}{1-j}\cdot3=\frac{3}{2}+j\frac{3}{2}
\end{gather*}

\subsection{Esercizio 2}
Calcolare l'espressione a regime della tensione di nodo $v_x$:
\begin{center}
    \begin{circuitikz}
        \draw (0,0) to[european voltage source=$10\cos(10t)$,i=$-i_a$](0,2)
                    to[R=$10\,\Omega$,-*](2,2)node[above]{$v_x$}
                    to[short](4,2)
                    to[european controlled voltage source=$10\,i_a$](6,2)
                    to[R=$5\,\Omega$](8,2)
                    to[L=$0.5\,H$](8,0)
                    to[short](0,0);
        \draw (3,0) to[short](3,-0.5)node[ground]{};
        \draw (2,0) to[R=$10\,\Omega$](2,2);
        \draw (4,0) to[C=$10\,mF$](4,2);
    \end{circuitikz}
\end{center}

Si risolve mediante il metodo dei nodi. Si considerano per ogni lato le loro impedenze:
\begin{gather*}
    z_1=z_2=10\,\Omega\\
    z_3=\displaystyle-\frac{j}{\omega C}=-\frac{j}{10\cdot 10\times10^{-3}}=-10j\,\Omega\\
    z_4=5\,\Omega+\omega L=5\,\Omega+5j\,\Omega
\end{gather*}
Poiché oltre al nodo di salto è presente un solo nodo, per il metodo dei nodi si ottiene un'unica equazione:
\begin{gather*}
    (y_1+y_2+y_3+y_4)\overline{V}_x=\overline{V}_{in}y_1-10\overline{I}_ay_4
\end{gather*}
Si esprime il vincolo del pilota:
\begin{gather*}
    \overline{I}_a=(\overline{V}_{in}-\overline{V}_x)y_1
\end{gather*}
Per cui l'equazione dei nodi diventa:
\begin{gather*}
    (y_1+y_2+y_3+y_4)\overline{V}_x=\overline{V}_{in}y_1-10(\overline{V}_{in}-\overline{V}_x)y_1y_4\\
    (y_1+y_2+y_3+y_4-10y_1y_4)\overline{V}_x=\overline{V}_{in}(y_1-10y_1y_4)\\
    \overline{V}_{x}=\displaystyle\frac{\overline{V}_{in}(y_1-10y_1y_4)}{(y_1+y_2+y_3+y_4-10y_1y_4)}
\end{gather*}
Si calcolano le ammettenze:
\begin{gather*}
    y_1=y_2=\displaystyle\frac{1}{10}\,\Omega^{-1}\\
    y_3=\displaystyle\frac{j}{10}\,\Omega^{-1}\\
    y_4=\displaystyle\frac{1}{5+5j}\,\Omega^{-1}=\frac{1}{10}\,\Omega^{-1}-\frac{j}{10}\,\Omega^{-1}
\end{gather*}
Da cui si ottiene un fasore di nodo:
\begin{gather*}
    \overline{V}_x=2+4i\\
    |\overline{V}_x|=\displaystyle\frac{10}{\sqrt{5}}\\
    \varphi=\arctan\left(\displaystyle\frac{4}{2}\right)\cdot\frac{180^{\circ}}{\pi}=63.4^{\circ}
\end{gather*}
Per cui l'espressione della tensione di nodo corrisponde a:
\begin{gather}
    v_x=|\overline{V}_x|\cos(\omega t+\varphi)=\displaystyle\frac{10}{\sqrt{5}}\cos(10t+63.4^{\circ})
\end{gather}


\end{document}