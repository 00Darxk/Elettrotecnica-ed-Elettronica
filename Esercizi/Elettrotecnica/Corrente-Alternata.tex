\documentclass{article}

\usepackage{cancel}
\usepackage{tikz}
\usepackage{amsmath}
\usepackage[includehead,nomarginpar]{geometry}
\usepackage{graphicx}
\usepackage{amsfonts} 
\usepackage{verbatim}
\usepackage{mathrsfs}  
\usepackage{lmodern}
\usepackage{braket}
\usepackage{bookmark}
\usepackage{circuitikz}
\usepackage[italian]{babel}
\usepackage{fancyhdr}
\usepackage{romanbarpagenumber}

\setlength{\headheight}{12.0pt}
\addtolength{\topmargin}{-12.0pt}

%% segnati con un doppio segno percentuale ("%%") le correzioni o inserimenti da apportare al testo

\hypersetup{
    colorlinks=true,
    linkcolor=black,
}

\renewcommand{\contentsname}{Indice}

\tikzset{block/.style = {draw, fill=white, very thick, rectangle, minimum height=1cm, minimum width=2cm},
         lblock/.style={draw,fill=white,very thick, rectangle, minimum height=3cm, minimum width=1cm},
         sum/.style= {draw, fill=white, very thick, circle, node distance=0.5cm}}


\title{Esercizi Svolti su Circuiti a Corrente Alternata}
\author{Giacomo Sturm}
\date{AA: 2023/2024 - Ing. Informatica}


\begin{document}

\pagenumbering{Roman}

\pagestyle{fancy}
\fancyhead{}\fancyfoot{}
\fancyhead[C]{Elettrotecnica ed Elettronica - Giacomo Sturm}
\fancyfoot[C]{\thepage}

\maketitle

\vspace{10mm}

\begin{center}
    Sorgente del file LaTeX disponibile su \url{https://github.com/00Darxk/Elettrotecnica-ed-Elettronica}
\end{center}

\clearpage

\tableofcontents

\clearpage

\pagenumbering{arabic}

\section{Esercizi Svolti il 24 Novembre}

\subsection{Esercizio 7.1}
Calcolare i valori di resistenza e capacità equivalente, del seguente circuito in regime sinusoidale alla pulsazione di $\omega=2\,\mbox{rad}/s$:
\begin{center}
    \begin{circuitikz}
        \draw (0,2)node[above]{$A$} to[R=$R_1$,*-] (2,2)  
                    to[L=$L_1$](4,2)
                    to[R=$R_2$](4,0)
                    to[short,-*](0,0)node[below]{$B$};
        \draw (2,2) to[C=$C_2$](2,0);
    \end{circuitikz}
\end{center}
\begin{gather*}
    C_2=0.5\,F\\
    L_1=1\,H\\
    R_1=R_2=1\,\Omega
\end{gather*}
Si può risolvere applicando trasformazioni in serie ed in parallelo dell'impedenza:
\begin{gather*}
    z_1=j\omega L_1+R_2=2j+1\\
    z_2=-j\displaystyle\frac{1}{\omega C_2}=-j\\
    z_3=R_1=1
\end{gather*}
Si considera il parallelo tra $z_1$ e $z_2$:
\begin{gather*}
    z_{p12}=\displaystyle\frac{z_1z_2}{z_1+z_2}=\frac{-j(1+2j)}{1+2j-j}=\frac{-j+2}{1+j}\cdot\frac{1-j}{1-j}\\
    z_{p12}=\frac{1-3j}{2}
\end{gather*}
Si considera la serie tra $z_{p12}$ e $z_3$:
\begin{gather*}
    z_{AB}=z_{p12}+z_3=\displaystyle\frac{1}{2}+j\frac{3}{2}+1=\frac{3}{2}+j\frac{3}{2}
\end{gather*}
\begin{gather}
    R_{eq}=\displaystyle\frac{3}{2}\,\Omega\\
    C_{eq}=\frac{2}{3}\,\Omega\cdot \omega=\frac{1}{3} F
\end{gather}

Alternativamente per trovare l'impedenza equivalente della rappresentazione Thevenin si considera un generatore che eroga una tensione $\overline{V}_{in}$, e si risolve 
mediante il metodo dei nodi o delle maglie:
\begin{center}
    \begin{circuitikz}
        \draw (0,2) to[R=$R_1$,*-] (2,2)node[above]{$A$} 
                    to[L=$L_1$](4,2)
                    to[R=$R_2$](4,0)
                    to[short,-*](0,0);
        \draw (2,2) to[C=$C_2$](2,0);
        \draw (0,0) to[short](-2,0)
                    to[sV=$\overline{V}_{in}$](-2,2)
                    to[short,i=$\overline{I}_{in}$](0,2);
        \draw (3,0)node[above]{$0$} to[short](3,-0.5)node[ground]{};
    \end{circuitikz}
\end{center}

La matrice delle ammettenze modali diventa solamente l'autoammettenza del nodo $A$:
\begin{gather*}
    (y_1+y_2+y_3)\overline{V}_A=\overline{V}_{in}y_3\\
    \overline{I}_{in}=\displaystyle\frac{\overline{V}_{in}-\overline{V}_A}{z_3}\\
    z_{in}=\displaystyle\frac{\overline{V}_{in}}{\overline{I}_{in}}
\end{gather*}
Si calcola ora numericamente:
\begin{gather*}
    \overline{V}_{in}=1\,V\\
    \overline{V}_A=\displaystyle\frac{y_3}{y_1+y_2+y_3}\cdot 1\,V\\
    \overline{V}_A=\displaystyle\frac{1}{\frac{1}{1+2j}-\frac{1}{j}+1}=\displaystyle\frac{1}{\frac{j-1-2j-j(1+2j)}{j(1+2j)}}=\frac{j(1+2j)}{j-1-2j+j-2}\\
    \overline{V}_A=\frac{j-2}{-3}\\
    \overline{I}_{in}=1+\displaystyle\frac{j-2}{3}=\frac{3+j-2}{3}=\frac{1+j}{3}\\
    z_{in}=\displaystyle\frac{1}{1+j}\cdot\frac{1-j}{1-j}\cdot3=\frac{3}{2}+j\frac{3}{2}
\end{gather*}

\subsection{Esercizio 7.2}
Calcolare l'espressione a regime della tensione di nodo $v_x$:
\begin{center}
    \begin{circuitikz}
        \draw (0,0) to[european voltage source=$10\cos(10t)$,i=$-i_a$](0,2)
                    to[R=$10\,\Omega$,-*](2,2)node[above]{$v_x$}
                    to[short](4,2)
                    to[european controlled voltage source=$10\,i_a$](6,2)
                    to[R=$5\,\Omega$](8,2)
                    to[L=$0.5\,H$](8,0)
                    to[short](0,0);
        \draw (3,0) to[short](3,-0.5)node[ground]{};
        \draw (2,0) to[R=$10\,\Omega$](2,2);
        \draw (4,0) to[C=$10\,mF$](4,2);
    \end{circuitikz}
\end{center}

Si risolve mediante il metodo dei nodi. Si considerano per ogni lato le loro impedenze:
\begin{gather*}
    z_1=z_2=10\,\Omega\\
    z_3=\displaystyle-\frac{j}{\omega C}=-\frac{j}{10\cdot 10\times10^{-3}}=-10j\,\Omega\\
    z_4=5\,\Omega+\omega L=5\,\Omega+5j\,\Omega
\end{gather*}
Poiché oltre al nodo di salto è presente un solo nodo, per il metodo dei nodi si ottiene un'unica equazione:
\begin{gather*}
    (y_1+y_2+y_3+y_4)\overline{V}_x=\overline{V}_{in}y_1-10\overline{I}_ay_4
\end{gather*}
Si esprime il vincolo del pilota:
\begin{gather*}
    \overline{I}_a=(\overline{V}_{in}-\overline{V}_x)y_1
\end{gather*}
Per cui l'equazione dei nodi diventa:
\begin{gather*}
    (y_1+y_2+y_3+y_4)\overline{V}_x=\overline{V}_{in}y_1-10(\overline{V}_{in}-\overline{V}_x)y_1y_4\\
    (y_1+y_2+y_3+y_4-10y_1y_4)\overline{V}_x=\overline{V}_{in}(y_1-10y_1y_4)\\
    \overline{V}_{x}=\displaystyle\frac{\overline{V}_{in}(y_1-10y_1y_4)}{(y_1+y_2+y_3+y_4-10y_1y_4)}
\end{gather*}
Si calcolano le ammettenze:
\begin{gather*}
    y_1=y_2=\displaystyle\frac{1}{10}\,\Omega^{-1}\\
    y_3=\displaystyle\frac{j}{10}\,\Omega^{-1}\\
    y_4=\displaystyle\frac{1}{5+5j}\,\Omega^{-1}=\frac{1}{10}\,\Omega^{-1}-\frac{j}{10}\,\Omega^{-1}
\end{gather*}
Da cui si ottiene un fasore di nodo:
\begin{gather*}
    \overline{V}_x=2+4i\\
    |\overline{V}_x|=\displaystyle\frac{10}{\sqrt{5}}\\
    \varphi=\arctan\left(\displaystyle\frac{4}{2}\right)\cdot\frac{180^{\circ}}{\pi}=63.4^{\circ}
\end{gather*}
Per cui l'espressione della tensione di nodo corrisponde a:
\begin{gather}
    v_x=|\overline{V}_x|\cos(\omega t+\varphi)=\displaystyle\frac{10}{\sqrt{5}}\cos(10t+63.4^{\circ})
\end{gather}

\clearpage

\section{Esercitazione Svolta il 29 Novembre}

\subsection{esercizio 7.3}

Determinare la tensione a regime dell'induttore $v_L$:
\begin{center}
    \begin{circuitikz}
        \draw (0,0) to[short] (4,0)
                    to[sV=$\overline{E}_{g2}$](4,2)
                    to[C=$C$](2,2)
                    to[R=$R$](2,0);
        \draw (0,0) to[sV=$\overline{E}_{g1}$](0,2)
                    to[L=$L$](2,2);
    \end{circuitikz}
\end{center}
\begin{gather*}
    E_{g1}=20\cos(1000 t)\,V\\
    E_{g2}=30\cos(1000t -90^{\circ})\,V\\
    R=10\,\Omega\\
    L=16\,mH\\
    C=200\,\mu F
\end{gather*}
Si determina la pulsazione e ed i fasori dei generatori:
\begin{gather*}
    \omega=10^3\,\mbox{rad}/s\\
    \overline{E}_{g1}=20\,V\\
    \overline{E}_{g2}=30e^{-j\pi/2}=-30j\,V
\end{gather*}
Si determinano le impedenze del circuito:
\begin{gather*}
    z_1=j\omega L=10^3\cdot15\cdot10^{-3}j\,\Omega=15e^{j\pi/2}\Omega\\
    z_2=R=10\,\Omega\\
    z_3=\displaystyle-\frac{j}{\omega C}=-j\frac{1}{10^3\cdot200\cdot10^{-6}}\,\Omega=5e^{-j\pi/2}\Omega
\end{gather*}
Si risolve mediante il metodo dei nodi, ottenendo una singola equazione, questo caso coincide con il teorema di Millman:
\begin{gather*}
    \overline{V}_A=\displaystyle\frac{y_1\overline{E}_{g1}+y_3\overline{E}_{g2}}{y_1+y_2+y_3}=\frac{\frac{20}{15j}+\frac{-30j}{-5j}}{\frac{1}{15j}+\frac{1}{10}+\frac{1}{-5j}}\\
    \displaystyle\frac{\frac{1}{5j}\left[\frac{20}{3}+30j\right]}{\frac{10-30+15j}{150j}}=\frac{30\left[\frac{20}{3}+30j\right]}{-20+15j}=\frac{30\left[\frac{20}{3}+30j\right]}{20^2+15^2}(-20-15j)\\
    \displaystyle-\frac{30}{625}\left(\frac{400}{3}+600j+100j-450\right)=15.2-33.6j\,V
\end{gather*}
Si determina ora la tensione ai capi dell'induttore:
\begin{gather*}
    \overline{V}_L=\overline{E}_{g1}-\overline{V}_A=20-15.2+33.6j\,V=4.8+33.6j\,V
\end{gather*}
Si determina ora la fase ed il modulo:
\begin{gather*}
    |V_L|=\sqrt{4.8^2+33.6^2}=33.941\\
    \varphi=\arctan\left(\displaystyle\frac{33.6}{4.8}\right)=81.2^{\circ}
\end{gather*}
Per cui l'espressione a regime della tensione corrisponde a:
\begin{gather}
    v_L(t)=33.941\cos(1000t+81.2^{\circ})\,V
\end{gather}

Si risolve ora mediante il metodo delle maglie:
\begin{gather*}
    \begin{bmatrix}
        z_1+z_2&-z_2\\-z_2&z_2+z_3
    \end{bmatrix}\begin{bmatrix}
        \overline{I}_{m_1}\\\overline{I}_{m_2}
    \end{bmatrix}=\begin{bmatrix}
        \overline{E}_{g1}\\
        -\overline{E}_{g2}
    \end{bmatrix}\\
    \begin{bmatrix}
        \overline{I}_{m_1}\\\overline{I}_{m_2}
    \end{bmatrix}=\begin{bmatrix}
        10+15j&-10\\-10&10-5j
    \end{bmatrix}^{-1}\begin{bmatrix}
        20\\-30j
    \end{bmatrix}=\begin{bmatrix}
        -1.6-3.2j\\1.2-5.6j
    \end{bmatrix}
\end{gather*}
Si calcola ora il fasore della tensione dell'induttore:
\begin{gather*}
    \overline{V}_L=\overline{E}_{g1}-(\overline{I}_{m_1}-\overline{I}_{m_2})z_1=4.8+33.6j\,V
\end{gather*}
Corrisponde allo stesso fasore calcolato precedentemente mediante il metodo dei nodi.

\subsection{Esercizio 7.4}
Determinare la potenza attiva erogata dal generatore:
\begin{center}
    \begin{circuitikz}
        \draw (6,0) to[short](0,0)
                    to[sV=$\overline{E}_g$](0,2)
                    to[R=$R_1$](2,2)
                    to[short](6,2)
                    to[C=$C$](6,0);
        \draw (2,2) to[R=$R_2$](2,0);
        \draw (4,2) to[L=$L$](4,0);
    \end{circuitikz}
\end{center}
\begin{gather*}
    E_g=5\cos(2t)\,V\\
    R_1=2\,\Omega\\
    R_2=1\,\Omega\\
    C=0.5\,F\\
    L=2H
\end{gather*}
Si calcolano le impedenze ed il fasore della tensione:
\begin{gather*}
    z_1=2\,\Omega\\
    z_2=1\,\Omega\\
    z_3=4j\,\Omega\\
    z_4=-j\,\Omega\\
    \overline{E}_g=5\,V
\end{gather*}

Si considera il parallelo tra $z_2$, $z_3$ e $z_4$:
\begin{gather*}
    y_p=\displaystyle1+\frac{1}{4j}+\frac{1}{-j}=\frac{4j+1-4}{4j}=\frac{-3+4j}{4j}\\
    z_p=\displaystyle\frac{4j}{-3+4j}=\frac{4j(-3-4j)}{25}=\frac{16}{25}-\frac{12}{25}j\,\Omega
\end{gather*}
Si applica il metodo delle maglie:
\begin{gather*}
    (z_1+z_p)\overline{I}_{m_1}=\overline{E}_g\\
    \overline{I}_{m_1}=\overline{I}_g=\displaystyle\frac{\overline{E}_g}{z_1+z_p}=\frac{5}{\frac{16}{25}-\frac{12}{25}j+2}\\
    \overline{I}_g=\displaystyle\frac{125}{66-12j}=\frac{125}{4500}(66+12j)\,A
\end{gather*}
La potenza complessa erogata dal generatore si ottiene come:
\begin{gather*}
    P_C=\displaystyle\frac{1}{2}\overline{E}_g\overline{I}_g^*=\frac{1}{2}5\frac{125}{4500}(66-12j)=4.5833-0.8333j
\end{gather*}

\subsection{Esercizio 7.6}

\begin{quotation}
    Un'impedenza $Z$ alimentata da un generatore sinusoidale da $120\,V$ efficaci, assorbe la potenza apparente di $12\,kVA$, con un fattore di potenza di $0.856$ induttivo. 
    Determinare modulo e fase di $Z$. 
\end{quotation}
Dato il fattore di potenza si può determinare la fase della potenza complessa:
\begin{gather*}
    \varphi\arccos(0.856)=31.1296^{\circ}
\end{gather*}

Si può esprimere la corrente che attraverso l'impedenza come il rapporto tra la potenza apparente e la tensione erogata:
\begin{gather*}
    \overline{I}=\displaystyle\frac{P_a}{\overline{E}_g}=\frac{12\cdot10^3}{120}=100\,A
\end{gather*}
Si ricorda che l'impedenza di un dato bipolo, tramite la legge di Ohm complessa, si può ricavare dal fasore della corrente e della tensione:
\begin{gather*}
    |Z|=\displaystyle\frac{\overline{E}_{g}}{\overline{I}_g}\frac{120\,V}{100\,A}=1.2\,\Omega
\end{gather*}

\subsection{Esercizio 7.7}

Determinare la potenza complessiva assorbita dall'impedenza $z_1$ composta dal resistore e dall'induttore in serie:
\begin{center}
    \begin{circuitikz}
        \draw (0,0) to[sV=$220V$](0,4)
                    to[R=$4\,\Omega$](2,4)
                    to[L=$2j\,\Omega$](4,4)
                    to[R=$15\,\Omega$](4,2)
                    to[C=$-10j\,\Omega$](4,0)
                    to[short](0,0);
    \end{circuitikz}
\end{center}

Si calcola il fasore della corrente:
\begin{gather*}
    \overline{I}=\displaystyle\frac{\overline{E}_g}{z_1+z_2}=\frac{220}{4+15+(2-10)j}=\frac{220}{425}(19+8j)
\end{gather*}

La potenza assorbita dall'impedenza $z_1$ si calcola come:
\begin{gather*}
    P_{C1}=z_1|\overline{I}|^2=(4+2j)\left|\frac{220}{425}(19+8j)\right|^2=455.4\,W+227.7j\,VA
\end{gather*}
Per cui questo circuito risulta ohmico-capacitivo, e per metterlo in risonanza bisognerebbe inserire un'induttore in parallelo. 

\subsection{Esercizio 7.21}

Determinare il circuito equivalente di Thevenin tra i terminali $A$ e $B$:
\begin{center}
    \begin{circuitikz}
        \draw (6,0)node[right]{$B$} to[short,*-](0,0)
                    to[sV=$\overline{E}_{g1}$](0,2)
                    to[short,i=$\overline{I}$](0.5,2)
                    to[R=$R$](2.5,2)
                    to[C=$C$](2.5,0);
        \draw (2.5,2) to[european controlled voltage source=$1/2\overline{I}$](4.5,2)
                    to[L=$L$](4.5,0);
        \draw (4.5,2) to[short,-*](6,2)node[right]{$A$};
    \end{circuitikz}
\end{center}
\begin{gather*}
    \overline{E}_g=5\,V\\
    z_R=1\,\Omega\\
    z_C=-3j\,\Omega\\
    z_L=2j\,\Omega
\end{gather*}
Si risolve mediante il metodo delle maglie:
\begin{gather*}
    \begin{bmatrix}
        z_R+z_C&-z_C\\-z_C&z_L
    \end{bmatrix}\begin{bmatrix}
        \overline{I}_{m_1}\\\overline{I}_{m_2}
    \end{bmatrix}=\begin{bmatrix}
        \overline{E}_{g1}\\\overline{E}_{g2}
    \end{bmatrix}\\
    \begin{bmatrix}
        1-3j&3j\\3j&-j
    \end{bmatrix}\begin{bmatrix}
        \overline{I}_{m_1}\\\overline{I}_{m_2}
    \end{bmatrix}=\begin{bmatrix}
        5\\\displaystyle\frac{1}{2}\overline{I}
    \end{bmatrix}
\end{gather*}
Il vincolo del pilota è banale, poiché la corrente di riferimento corrisponde alla corrente della maglia $1$:
\begin{gather*}
    \overline{I}=\overline{I}_{m_1}\\
    \begin{bmatrix}
        1-3j&+3j\\\displaystyle-\frac{1}{2}+3j&-j
    \end{bmatrix}\begin{bmatrix}
        \overline{I}_{m_1}\\\overline{I}_{m_2}
    \end{bmatrix}=\begin{bmatrix}
        5\\0
    \end{bmatrix}\\
    \begin{bmatrix}
        \overline{I}_{m_1}\\\overline{I}_{m_2}
    \end{bmatrix}=\begin{bmatrix}
        1-3j&+3j\\\displaystyle-\frac{1}{2}+3j&-j
    \end{bmatrix}^{-1}\begin{bmatrix}
        5\\0
    \end{bmatrix}=\begin{bmatrix}
        -0.069-0.8276j\\0.2069-2.5172j
    \end{bmatrix}
\end{gather*}
La tensione di Thevenin equivalente si ottiene come:
\begin{gather*}
    \overline{E}_{th}=z_L\overline{I}_{m_2}=2j(0.2069-2.5172j)=5.0345+0.4138j\,V
\end{gather*}
Si rappresenta in coordinate polari:
\begin{gather*}
    |\overline{E}_{th}|=\sqrt{5.0345^2+0.4138^2}=5.0515\\
    \varphi=\arctan\left(\displaystyle\frac{0.4138}{5.0345}\right)=4.6897^{\circ}
\end{gather*}
\begin{equation}
    \overline{E}_g=5.0515e^{j4.6897^{\circ}}
\end{equation}
Applicando il metodo dei nodi si ottiene una risposta analoga:
\begin{gather*}
    \overline{V}_A=\displaystyle\frac{\overline{E}gz_R-\frac{1}{2}\overline{I}y_L}{y_R+y_C+y_L}
\end{gather*}
Dove l'equazione del pilota in questo caso si ottiene come:
\begin{gather*}
    \overline{I}=(\overline{E}_g-\overline{V}_A)y_R
\end{gather*}
Per cui si ottiene un'equazione per la tensione:
\begin{gather*}
    \overline{V}_A=\displaystyle\frac{\overline{E}gz_R}{y_R+y_C+y_L}-\frac{1}{2}y_Ly_R\frac{\overline{E}_g}{y_R+y_C+y_L}+\frac{1}{2}y_LyR\frac{\overline{V}_A}{y_R+y_C+y_L}\\
    \overline{V}_A\left(\displaystyle1-\frac{1}{2}\frac{y_Ry_L}{y_R+y_C+y_L}\right)=\frac{\overline{E}gz_R-\frac{1}{2}y_Ly_R\overline{E}_g}{y_R+y_C+y_L}
\end{gather*}

Per calcolare l'impedenza di Thevenin si spengono solo i generatori passivi:
\begin{center}
    \begin{circuitikz}
        \draw (6,0)node[below]{$B$} to[short,*-](0,0)
                    to[short](0,2)
                    to[short,i=$\overline{I}$](0.5,2)
                    to[R=$R$](2.5,2)
                    to[C=$C$](2.5,0);
        \draw (2.5,2) to[european controlled voltage source=$1/2\overline{I}$](4.5,2)
                    to[L=$L$](4.5,0);
        \draw (4.5,2) to[short,-*](6,2)node[above]{$A$};
        \draw (6.1,0) to[short](8,0)
                    to[sV=$\overline{E}_C$](8,2)
                    to[short,i=$\overline{I}_C$](6.1,2);
    \end{circuitikz}
\end{center}
Si inserisce dall'esterno un generatore indipendente in base al metodo di risoluzione, per cui in questo caso si inserisce un generatore di tensione esterno per applicare il 
metodo delle maglie, che eroga $\overline{E}_C=1\,V$. L'induttore non influisce sulla tensione in ingresso, per cui si può calcolare prima un sistema di due equazioni, per 
trovare due correnti di maglia $\overline{I}_{m_1}$ e $\overline{I}_{m_2}$: 
\begin{gather*}
    \begin{bmatrix}
        z_R+z_C&-z_C\\-z_C&z_C
    \end{bmatrix}\begin{bmatrix}
        \overline{I}_{m_1}\\\overline{I}_{m_2}
    \end{bmatrix}=\begin{bmatrix}
        0\\\displaystyle\frac{1}{2}\overline{I}-\overline{E}_C
    \end{bmatrix}\\
    \begin{bmatrix}
        z_R+z_C&-z_C\\-\displaystyle\frac{1}{2}-z_C&z_C
    \end{bmatrix}\begin{bmatrix}
        \overline{I}_{m_1}\\\overline{I}_{m_2}
    \end{bmatrix}=\begin{bmatrix}
        0\\-\overline{E}_C
    \end{bmatrix}\\
    \begin{bmatrix}
        \overline{I}_{m_1}\\\overline{I}_{m_2}
    \end{bmatrix}=\begin{bmatrix}
        1-3j&+3j\\\displaystyle-\frac{1}{2}+3j&-3j
    \end{bmatrix}^{-1}\begin{bmatrix}
        0\\-1
    \end{bmatrix}=\begin{bmatrix}
        -2\\
        -2-\displaystyle\frac{2}{3}j
    \end{bmatrix}
\end{gather*}

La corrente che si vuole misurare corrisponde alla corrente della seconda maglia sommata alla corrente 
attraverso l'induttore $\overline{I}_C=\overline{I}_{m_1}+\overline{I}_L=\overline{I}_{m_1}+\frac{\overline{E}_C}{z_L}$:
\begin{gather*}
    \overline{I}_C=2+\displaystyle\frac{2}{3}j-\frac{1}{2}j=2-\frac{1}{6}j
\end{gather*}
L'impedenza equivalente si ottiene dal rapporto tar la tensione inserita nel circuito e la corrente calcolata:
\begin{gather*}
    z_{eq}=\displaystyle\frac{\overline{E}_C}{\overline{I}_C}=\frac{1}{2-\frac{1}{6}j}=\frac{1}{4+\frac{1}{36}}\left(2-\frac{1}{6}j\right)=\frac{36}{145}\left(2-\frac{1}{6}j\right)
\end{gather*}
Rappresentandola in modulo e fase:
\begin{gather*}
    |z_{eq}|=\displaystyle\frac{36}{145}\sqrt{4+\frac{1}{6}}=0.4983\\
    \varphi=\arctan\left(\frac{-\frac{1}{6}}{2}\right)=-4.7636^{\circ}
\end{gather*}
Per cui l'impedenza equivalente della rappresentazione esterna di Thevenin risulta essere:
\begin{gather}
    z_{eq}=0.4983e^{-j4.7636^{\circ}}
\end{gather}

\clearpage 

\section{Esercitazione Svolta il 30 Novembre}

\subsection{Esercizio 7.8}

Determinare l'ammettenza equivalente del seguente circuito:
\begin{center}
    \begin{circuitikz}
        \draw (0,2) to[short,i=$\overline{I}_1$,*-] (0.5,2)
                    to[R=$R$](2.5,2)
                    to[L=$L$](2.5,0)
                    to[short,-*](0,0);
        \draw (2.5,0) to[short](5,0)
                    to[european controlled voltage source=$0.5\overline{I}_1$](5,2)
                    to[C=$C$](2.5,2);
    \end{circuitikz}
\end{center}
\begin{gather*}
    |z_R|=0.5\,\Omega\\
    |z_C|=0.25\,\Omega\\
    |z_L|=1\,\Omega
\end{gather*}

Si inserisce un generatore di tensione esterno $\overline{E}_e=1\,V$. 
Si applica quindi il metodo delle maglie:
\begin{gather*}
    \begin{bmatrix}
        z_R+z_L&-z_C\\-z_C&z_C+z_L
    \end{bmatrix}\begin{bmatrix}
        \overline{I}_{m_1}\\\overline{I}_{m_2}
    \end{bmatrix}=\begin{bmatrix}
        \overline{E}_e\\-0.5\overline{I}_1
    \end{bmatrix}=\begin{bmatrix}
        \overline{E}_e\\0
    \end{bmatrix}+\begin{bmatrix}
        0\\-0.5\overline{I}_1
    \end{bmatrix}
\end{gather*}
Si può notare come la corrente di lato $\overline{I}_1$ coincide con la corrente di maglia $\overline{I}_{m_1}$, per cui l'equazione diventa:
\begin{gather*}
    \begin{bmatrix}
        z_R+z_L&-z_C\\-z_C+0.5&z_C+z_L
    \end{bmatrix}\begin{bmatrix}
        \overline{I}_{m_1}\\\overline{I}_{m_2}
    \end{bmatrix}=\begin{bmatrix}
        \overline{E}_e\\0
    \end{bmatrix}\\
    \begin{bmatrix}
        \overline{I}_{m_1}\\\overline{I}_{m_2}
    \end{bmatrix}=\displaystyle\frac{1}{(z_R+z_L)(z_C+z_L)-(-z_C+0.5)(-z_C)}\begin{bmatrix}
        z_C+z_L&z_C-0.5\\z_C&z_R+z_L
    \end{bmatrix}\begin{bmatrix}
        \overline{E}_e\\0
    \end{bmatrix}
\end{gather*}
Dopo aver calcolato la corrente di maglia $\overline{I}_{m_1}$, si esprime l'ammettenza equivalente come:
\begin{gather}
    y_{eq}=\displaystyle\frac{\overline{I}_{m_1}}{\overline{E}_e}=0.792+j0.226\,\Omega^{-1}
\end{gather}


Alternativamente si può calcolare tramite il metodo dei Nodi, o per il teorema di Millman:
\begin{gather*}
    \overline{V}_A=\displaystyle\frac{\overline{E}_ey_R+0.5\overline{I}_1y_C}{y_R+y_C+y_L}
\end{gather*}
Si inserisce il vincolo del pilota come combinazione lineare dei potenziali modali:
\begin{gather*}
    \overline{I}_1=(\overline{E}_e-\overline{V}_A)y_R
\end{gather*}
Per cui l'equazione diventa:
\begin{gather*}
    \overline{V}_A=\displaystyle\frac{\overline{E}_ey_R+0.5(\overline{E}_e-\overline{V}_A)y_Ry_C}{y_R+y_C+y_L}\\
    \overline{V}_A\left(1+\displaystyle\frac{0.5y_Ry_C}{y_R+y_C+y_L}\right)=\frac{\overline{E}_ey_R+0.5\overline{E}_ey_Ry_C}{y_R+y_C+y_L}\\
    \overline{V}_A=\displaystyle\frac{\frac{\overline{E}_ey_R+0.5\overline{E}_ey_Ry_C}{y_R+y_C+y_L}}{1+\frac{0.5y_Ry_C}{y_R+y_C+y_L}}
\end{gather*}
L'ammettenza equivalente si ottiene come:
\begin{gather}
    y_{eq}=\displaystyle\frac{\overline{I}_1}{\overline{V}_A}=\frac{\overline{E}_e-\overline{V}_A}{\overline{V}_A}y_R=0.792+j0.226\,\Omega^{-1}
\end{gather}


Invece inserendo un generatore di corrente $\overline{I}_g=1\,A$, allora in questo caso assume un valore di $0.5\,V$. Per cui in questo caso il potenziale nodale di $A$ si 
calcola come:
\begin{gather*}
    \overline{V}_A=\displaystyle\frac{1+0.5y_3}{y_2+y_3}
\end{gather*}
Calcolando l'impedenza equivalente si considera anche in serie l'impedenza $z_R$:
\begin{gather*}
    z_{eq}=\displaystyle\frac{\overline{V}_A}{\overline{I}_g}+z_R
\end{gather*}
Per cui l'ammettenza si ottiene come il reciproco di quest'impedenza così calcolata:
\begin{gather*}
    y_{eq}=\displaystyle\frac{1}{z_{eq}}
\end{gather*}

\subsection{Esercizio 7.9}

Determinare la rappresentazione equivalente di Thevenin del seguente circuito:

\begin{center}
    \begin{circuitikz}
        \draw (4,0) to[short,*-](0,0)
                    to[european voltage source=$\overline{E}$](0,2)
                    to[L=$L$](2,2)
                    to[R=$R$,i=$\overline{I}_0$,-*](4,2);
        \draw (2,0) to[european controlled current source=$3\overline{I}_0$](2,2);
    \end{circuitikz}
\end{center}

Per calcolare la tensione di Thevenin, poiché il resistore è collegato a vuoto, la corrente pilota è nulla $\overline{I}_0=0\,A$, per cui la tensione di Thevenin equivalente 
corrisponde alla tensione del generatore montato in parallelo:
\begin{gather*}
    \overline{E}_{th}=1+j0\,V
\end{gather*}
Si rende passiva la rete e si inserisce un generatore di tensione in entrata $\overline{I}_g=1\,A$, in modo che la corrente pilotata coincida con la corrente inserita:
\begin{gather*}
    \overline{I}_0=\overline{I}_g
\end{gather*}
Si considera il metodo dei nodi:
\begin{gather*}
    \overline{V}_A=\displaystyle\frac{3\overline{I}_0-\overline{I}_0}{y_L}=j4\,V
\end{gather*}

Per cui l'impedenza equivalente si ottiene come:
\begin{gather}
    z_{eq}=\displaystyle\frac{\overline{V}_A}{\overline{I}_g}+1=1-j4
\end{gather}

\subsection{Esercizio 7.10}

\end{document}