\documentclass{article}

\usepackage{cancel}
\usepackage{tikz}
\usepackage{amsmath}
\usepackage[includehead,nomarginpar]{geometry}
\usepackage{graphicx}
\usepackage{amsfonts} 
\usepackage{verbatim}
\usepackage{mathrsfs}  
\usepackage{lmodern}
\usepackage{braket}
\usepackage{bookmark}
\usepackage{circuitikz}
\usepackage[italian]{babel}
\usepackage{fancyhdr}
\usepackage{romanbarpagenumber}

\setlength{\headheight}{12.0pt}
\addtolength{\topmargin}{-12.0pt}


\hypersetup{
    colorlinks=true,
    linkcolor=black,
}

\renewcommand{\contentsname}{Indice}

\tikzset{block/.style = {draw, fill=white, very thick, rectangle, minimum height=1cm, minimum width=2cm},
         lblock/.style={draw,fill=white,very thick, rectangle, minimum height=3cm, minimum width=1cm},
         sum/.style= {draw, fill=white, very thick, circle, node distance=0.5cm}}


\title{Esercizi Svolti sul Dominio di Laplace}
\author{Giacomo Sturm}
\date{AA: 2023/2024 - Ing. Informatica}


\begin{document}

\pagenumbering{Roman}

\pagestyle{fancy}
\fancyhead{}\fancyfoot{}
\fancyhead[C]{Elettrotecnica ed Elettronica - Giacomo Sturm}
\fancyfoot[C]{\thepage}

\maketitle

\vspace{10mm}

\begin{center}
    Sorgente del file LaTeX disponibile su \url{https://github.com/00Darxk/Elettrotecnica-ed-Elettronica}
\end{center}

\clearpage

\tableofcontents

\clearpage

\pagenumbering{arabic}

\section{Esercizi Svolti l'11 Dicembre}

\subsection{Esercizio 2.5}

Calcolare la tensione ai capi dei morsetti $v_o(t)$ del seguente circuito, dopo la commutazione da $A$ e $B$: 

\begin{center}
    \begin{circuitikz}
        \draw (4.5,0)node[above]{$-$}to[short,*-](0,0)
                    to[european voltage source=$12\,V$](0,2)
                    to[short,-*](0.75,2)node[above]{$A$};
        \draw (1,0) to[short,-*](1,1.75)node[right]{$B$};
        \draw (1.25,2) to[L=$0.5\,H$,*-](3,2);
        \draw (3,0) to[R=$0.2\,\Omega$](3,2);
        \draw (3,2) to[short](4,2)
                    to[C=$3\,F$](4,0);
        \draw (4,2) to[short,-*](4.5,2)node[below]{$+$};
        \draw[dashed](0.75,2)--(1.25,2);
        \draw[-](1.25,2)--(1,1.75);
    \end{circuitikz}
\end{center}

Il circuito prima della commutazione si trova a regime, per cui la tensione ai capi del condensatore corrisponde alla tensione erogata dal generatore:
\begin{gather*}
    v_C(0^-)=12\,V
\end{gather*}
La memoria dell'induttore invece corrisponde a:
\begin{gather*}
    i(0^-)=\displaystyle\frac{12\,V}{0.2\,\Omega}=60\,A
\end{gather*}

\begin{center}
    \begin{circuitikz}
        \draw (0,0) to[short] (6,0) 
                    to[european voltage source=$\frac{v_C(0^-)}{s}$](6,2)
                    to[generic=$\frac{1}{sC}$](6,4)
                    to[short](4,4)
                    to[generic=$0.2\,\Omega$](4,0);
        \draw (0,0) to[short](0,4)
                    to[generic=$sL$](2,4)
                    to[european voltage source=$Li(0^-)$,-*](4,4);
        \draw (4,0) to[short,*-](4,-0.5)node[ground]{};
    \end{circuitikz}
\end{center}
Si applica il metodo dei nodi, corrispondente al teorema di Millman in questo caso:
\begin{gather*}
    V_{out}(s)=\displaystyle\frac{\displaystyle\frac{Li(0^-)}{sL}+\frac{sCv_C(0^-)}{s}}{\displaystyle\frac{1}{sL}+\frac{1}{R}+sC}=\frac{\displaystyle\frac{Li(0^-)+sLCv_C(0^-)}{sL}}{\displaystyle\frac{1+sGL+s^2LC}{sL}}\\
    V_{out}(s)=\displaystyle\frac{Li(0^-)+sLCv_C(0^-)}{LC\left(s^2+s\displaystyle\frac{1}{RC}+\frac{1}{LC}\right)}
\end{gather*}
Si considera il denominatore e si individuano i poli $s_1$ e $s_2$:
\begin{gather*}
    D(s)=LC\left(s^2+s\displaystyle\frac{1}{RC}+\frac{1}{LC}\right)=LC(s-s_1)(s-s_2)
\end{gather*}
Si considerano i valori delle grandezze:
\begin{gather*}
    Li(0^-)=30\\
    LC=\displaystyle\frac{3}{2}\\
    \displaystyle\frac{1}{RC}=\frac{5}{3}
\end{gather*}
Sostituendo i valori ottenuti nella funzione, si ottiene:
\begin{gather*}
    \displaystyle D(s)=\frac{3}{2}\left[s^2+\frac{5}{3}s+\frac{2}{3}\right]\\
    s_{1,2}=\displaystyle\frac{1}{2}\left(-\frac{5}{3}\pm\sqrt{\frac{25}{9}-\frac{8}{3}}\right)=-\frac{5}{6}\pm\sqrt{\frac{25-24}{9}}=-\frac{5}{6}\pm\frac{1}{6}\\
    s_1=-1\\
    s_2=\displaystyle-\frac{2}{3}\\
    F(s)=\displaystyle\frac{30+s\displaystyle\frac{3}{2}12}{\displaystyle\frac{3}{2}(s+1)\left(s+\frac{2}{3}\right)}
\end{gather*}
Si scompone la funzione di rete in fratti semplici:
\begin{gather*}
    F(s)=\displaystyle\frac{A_1}{s-s_1}+\frac{A_2}{s-s_2}=\frac{20+12s}{(s+1)\left(s+\displaystyle\frac{2}{3}\right)}
\end{gather*}
Si considerano due metodi per risolvere:
\begin{gather*}
    A_1(s-s_1)+A_2(s-s_2)=20+12s\\
    \begin{cases}
        A_1+A_2=12\\
        A_1s_1+A_2s_2=-20
    \end{cases}\\
    f(t)=A_1e^{s_1t}+A_2e^{s_2t}
\end{gather*}
Poiché i poli hanno parte reale negativa, per cui la funzione di rete è asintoticamente stabile. 
Alternativamente si applica il metodo dei residui per determinare $A_1$ e $A_2$:
\begin{gather*}
    \lim_{s\to s_1}(s-s_1)F(s)=\lim_{s\to s_1}\displaystyle\frac{20+12s}{s-s_2}=\lim_{s\to s_1}\left(A_1+A_2\frac{s-s_1}{s-s_2}\right)=A_1\\
    A_1=\displaystyle\frac{20+12s_1}{s_1-s_2}=\frac{20-12}{-1+\displaystyle\frac{2}{3}}=-24\\
    A_2=\displaystyle\frac{20+12s_2}{s_2-s_1}=\frac{20-8}{-\displaystyle\frac{2}{3}+1}=36
\end{gather*}
Per cui la funzione nel dominio del tempo risulta essere:
\begin{gather*}
    f(t)=-24e^{-t}+36^{-\frac{2}{3}t}=v_o(t)
\end{gather*}
Corrisponde alla tensione ai capi del condensatore, e coincide alla tensione ai capi dei morsetti del circuito, poiché la commutazione avviene in un tempo $t=0$, questa 
espressione è valida solo per un tempo $t\geq0^+$:
\begin{equation}
    v_o(t)=(36e^{-\frac{2}{3}t}-24e^{-t})u_{-1}(t)
\end{equation}

\section{Esercizi Svolti l'8 Gennaio}

\subsection{Esercizio 2.7}

Determinare l'espressione della corrente $i_L(t)$, dopo la commutazione, del seguente circuito:
\begin{center}
    \begin{circuitikz}
        \draw (0,0)to[european voltage source=$12\,V$](0,2)node[above right]{${t=0}$}
        to[closing switch](2,2)
        to[R=$1\,\Omega$](4,2)
        to[R=$1\,\Omega$](6,2)
        to[L=$1\,H$,i=$i_L(t)$](6,0)
        to[short](0,0)
        (4,0)to[C=$1\,F$](4,2);
    \end{circuitikz}
\end{center}
\begin{gather*}
    i_L(0^-)=4\,A\\
    v_C(0^-)=8\,V
\end{gather*}

Si rappresenta il circuito equivalente nel dominio di Laplace:
\begin{center}
    \begin{circuitikz}
        \draw (0,0) to[european voltage source=$\displaystyle\frac{12}{s}$](0,4)
        to[R=$1$](2,4)
        (0,0)to[short](2,0)
        to[european voltage source=$\displaystyle\frac{8}{s}$](2,2)
        to[C=$-\displaystyle\frac{1}{s}$](2,4)
        to[R=$1$](4,4)
        to[L=$s$](4,2)
        to[european voltage source=$4$](4,0)
        to[short](2,0);
        \draw[->](2.5,0.25)--(2.5,3.75)node[midway, left]{$V_C(s)$};
    \end{circuitikz}
\end{center}
Si risolve mediante il metodo dei nodi per avere direttamente la trasformata di $i_L(t)$:
\begin{gather*}
    \begin{bmatrix}
        1+\displaystyle\frac{\strut 1}{\strut s}&\displaystyle-\frac{\strut 1}{\strut s}\\ \displaystyle-\frac{\strut 1}{\strut s}&\displaystyle\frac{\strut 1}{\strut s}+s+1
    \end{bmatrix}\begin{bmatrix}
        I_{m_1}(s)\\I_{m_2}(s)
    \end{bmatrix}=\begin{bmatrix}
        \displaystyle\frac{\strut 12}{\strut s}-\frac{\strut 8}{\strut s}\\ \displaystyle\frac{\strut 8}{\strut s}+4
    \end{bmatrix}\\
    \begin{bmatrix}
        \displaystyle\frac{\strut s+1}{\strut s}&\displaystyle-\frac{\strut 1}{\strut s}\\ \displaystyle-\frac{\strut 1}{\strut s}&\displaystyle\frac{\strut 1+s(s+1)}{\strut s}
    \end{bmatrix}\begin{bmatrix}
        I_{m_1}(s)\\I_{m_2}(s)
    \end{bmatrix}=\begin{bmatrix}
        \displaystyle\frac{\strut 4}{\strut s}\\ \displaystyle\frac{\strut 4(s+2)}{\strut s}
    \end{bmatrix}
\end{gather*}
Si ricava la corrente della seconda maglia tramite il metodo di Cramer:
\begin{gather*}
    I_{m_2}(s)=\displaystyle\frac{s^2}{(s+1)+s(s+1)^2-1}\begin{vmatrix}
        \displaystyle\frac{\strut s+1}{\strut s}&\displaystyle\frac{\strut 4}{\strut s}\\ \displaystyle-\frac{\strut 1}{\strut s}
        &\displaystyle\frac{\strut 4(s+2)}{\strut s}
    \end{vmatrix}\\
    I_{m_2}(s)=\displaystyle\frac{s^2}{(s+1)+s(s+1)^2-1}\frac{4+(8s+8+4s^2+4s)}{s^2}=\frac{4s^2+12s+12}{s(s^2+2s^1+2)}=I_L(s)
\end{gather*}

Si individuano ora i poli al finito di $I_L(s)$:
\begin{gather*}
    s_{p_1}=0\\
    s_{p_{2,3}}=-{-1\pm\sqrt{1-2}}=-1\pm j\\
    s_{p_2}=-1+j=s_{p_3}^*
\end{gather*}
Si calcolano i residui $A$ e $B$, poiché il residuo di $s_{p_3}$ è pari al coniugato del suo coniugato $B^*$, usando la formula per i residui:
\begin{gather*}
    I_L(s)=\displaystyle\frac{A}{s-s_{p_1}}+\frac{B}{s-s_{p_2}}+\frac{B^*}{s-s_{p_2}^*}\\
    A=\lim_{s\to 0}\displaystyle\cancel{\frac{s}{s}}\frac{4s^2+12s+12}{s^2+2s+2}=\frac{12}{2}=6\\
    B=\lim_{s\to -1+j}\displaystyle\cancel{\frac{s+1-j}{s+1-j}}\frac{4s^2+12s+12}{s(s+1+j)}=\frac{4(-1+j)^2+12(-1+j)+12}{(-1+j)(-1+j+1+j)}\\
    \displaystyle\frac{1}{2j}\frac{4-4-8j-12+12j+12}{-1+j}=\frac{1}{2j}\frac{4j}{-1+j}\frac{-1-j}{-1-j}=\frac{1}{2j}\frac{4j(-1-j)}{2}\\    
    \displaystyle\frac{1}{2j}(-2j+2)=\frac{1}{2j}\left(2\sqrt{2}e^{-j\frac{\pi}{4}}\right)
\end{gather*}
Ricordando l'espressione in fasori di un'entrata sinusoidale:
\begin{gather*}
    I_M\sin(\omega t+\beta)=\displaystyle\frac{I_M}{2j}e^{j\beta}e^{\omega t}-\frac{I_M}{2j}e^{-j\beta}e^{-\omega t}
\end{gather*}
La trasformata di Laplace di un esponenziale si esprime come:
\begin{gather*}
    \mathcal{L}\{Ae^{\sigma t}\}=\displaystyle\frac{A}{s-\sigma}
\end{gather*}
Per cui la trasformata di Laplace di una sinusoide smorzata si ottiene come:
\begin{gather*}
    \mathcal{L}\{e^{j\sigma t}I_M\sin(\omega t+\beta)\}=\mathcal{L}\left\{e^{j\sigma t}\left[\displaystyle\frac{I_M}{2j}e^{j\beta}e^{j\omega t}-\frac{I_M}{2j}e^{-j\beta}e^{-j\omega t}\right]\right\}\\
    \mathcal{L}\left\{\displaystyle\frac{I_M}{2j}e^{j\beta}e^{(j\omega+\sigma) t}-\frac{I_M}{2j}e^{-j\beta}e^{-(j\omega+\sigma) t}\right\}=\frac{I_M}{2j}e^{\alpha t}\frac{1}{s-(j\omega +\sigma)}-\frac{I_M^*}{2j}e^{-\alpha t}\frac{1}{s-(-j\omega +\sigma)}
\end{gather*}
I residui ottenuti da $I_L(s)$ corrispondono quindi ad una funzione sinusoidale nel domino del tempo:
\begin{gather*}
    I_L(s)=\displaystyle\frac{6}{s}+\frac{2\sqrt{2}e^{-j\frac{\pi}{4}}}{2j}\frac{1}{s-(-1+j)}-\frac{2\sqrt{2}e^{j\frac{\pi}{4}}}{2j}\frac{1}{s-(-1-j)}\\
    i_L(t)=6+2\sqrt{2}^{-t}\sin\left(\omega-\displaystyle\frac{\pi}{4}\right)
\end{gather*}
Poiché il circuito si chiude per $t=0$, si moltiplica l'espressione ottenuta per il segnale gradino:
\begin{equation}
    i_L(t)=\left[6+2\sqrt{2}^{-t}\sin\left(\omega-\displaystyle\frac{\pi}{4}\right)\right]u(t)
\end{equation}



\end{document}