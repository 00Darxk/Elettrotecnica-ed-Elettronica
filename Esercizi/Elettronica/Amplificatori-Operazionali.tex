\documentclass{article}

\usepackage{cancel}
\usepackage{tikz}
\usepackage{amsmath}
\usepackage[includehead,nomarginpar]{geometry}
\usepackage{graphicx}
\usepackage{amsfonts} 
\usepackage{verbatim}
\usepackage{mathrsfs}  
\usepackage{lmodern}
\usepackage{braket}
\usepackage{bookmark}
\usepackage{circuitikz}
\usepackage[italian]{babel}
\usepackage{fancyhdr}
\usepackage{romanbarpagenumber}

\setlength{\headheight}{12.0pt}
\addtolength{\topmargin}{-12.0pt}


\hypersetup{
    colorlinks=true,
    linkcolor=black,
}

\renewcommand{\contentsname}{Indice}

\title{Esercizi Svolti su Amplificatori Operazionali}
\author{Giacomo Sturm}
\date{AA: 2023/2024 - Ing. Informatica}


\begin{document}

\pagenumbering{Roman}

\pagestyle{fancy}
\fancyhead{}\fancyfoot{}
\fancyhead[C]{Elettrotecnica ed Elettronica - Giacomo Sturm}
\fancyfoot[C]{\thepage}

\maketitle

\vspace{10mm}

\begin{center}
    Sorgente del file LaTeX disponibile su \url{https://github.com/00Darxk/Elettrotecnica-ed-Elettronica}
\end{center}

\clearpage

\tableofcontents

\clearpage

\pagenumbering{arabic}

\section{Esercizi Svolti il 10 Gennaio}

\subsection{Esercizio 1.1}

Determinare la tensione in uscita $V_o$ dal seguente circuito:
\begin{center}
    \begin{circuitikz}[voltage dir=old]
        \draw (0,0) node[op amp](amp){}
        (amp.down)to[short]++(0,-1)node[ground]{}
        (amp.out)to[short,-d](3,0)node[right]{$O$}
        to[short,-o](3,-0.5)node[right]{$+$}node[below]{$v_o$}
        (3,-2)node[right]{$-$}to[short,o-](3,-2.25)node[tlground]{}
        (-2,-2.5)node[ground]{}to[european voltage source=$V_2$](-2,-0.5)
        to[short](amp.+)
        (-4,-1.5)node[ground]{}to[european voltage source=$V_1$](-4,0.5)
        to[R=$R_1$,-d](-1,0.5)node[above left]{$A$}
        to[short](amp.-)
        (-1,0.5)to[short](-1,2.5)
        to[R=$R_2$,-d](1,2.5)node[above]{$B$}
        to[R=$R_4$](1,1)node[ground]{}
        (1,2.5)to[R=$R_3$](3,2.5)
        to[short](3,0);
    \end{circuitikz}
\end{center}

Il generatore $V_2$ connesso all'ingresso non invertente del circuito non verrà considerato nel calcolo dei nodi, poiché il suo contributo in corrente è 
nullo. 
Per la legge costitutiva di un operazionale ideale si ricava l'equazione di vincolo, ovvero il nodo $A$, all'ingresso invertente, si trova allo 
stesso potenziale dell'ingresso non invertente, ovvero ha una tension $V_2$, nota. 
Si risolve mediante il metodo dei nodi:
\begin{gather*}
    \begin{bmatrix}
        \displaystyle\frac{\strut 1}{\strut R_1}+\frac{\strut 1}{\strut R_2}&-\displaystyle\frac{\strut 1}{\strut R_2}&0\\
        -\displaystyle\frac{\strut 1}{\strut R_2}&\displaystyle\frac{\strut 1}{\strut R_2}+\frac{\strut 1}{\strut R_3}+\frac{\strut 1}{\strut R_4}&-\displaystyle\frac{\strut 1}{\strut R_3}\\
        0&-\displaystyle\frac{\strut 1}{\strut R_3}&\displaystyle\frac{\strut 1}{\strut R_3}
    \end{bmatrix}\begin{bmatrix}
        V_A=V_2\\V_B\\V_O
    \end{bmatrix}=\begin{bmatrix}
        V_1\displaystyle\frac{\strut 1}{\strut R_1}\\0\\-i_x
    \end{bmatrix}\\
    \begin{bmatrix}
        0&-\displaystyle\frac{\strut 1}{\strut R_2}&0\\
        0&\displaystyle\frac{\strut 1}{\strut R_2}+\frac{\strut 1}{\strut R_3}+\frac{\strut 1}{\strut R_4}&-\displaystyle\frac{\strut 1}{\strut R_3}\\
        1&-\displaystyle\frac{\strut 1}{\strut R_3}&\displaystyle\frac{\strut 1}{\strut R_3}
    \end{bmatrix}\begin{bmatrix}
        i_x\\V_B\\V_O
    \end{bmatrix}=\begin{bmatrix}
        -\displaystyle\frac{\strut 1}{\strut R_1}-\frac{\strut 1}{\strut R_2}V_2+\frac{\strut 1}{\strut R_1}V_1\\
        \displaystyle\frac{\strut 1}{\strut R_2}V_2\\
        0
    \end{bmatrix}
\end{gather*}
Si applica il metodo di Cramer per ottenere la tensione in uscita:
\begin{gather*}
    V_o=R_2R_3\begin{vmatrix}
        0 &-\displaystyle\frac{\strut1}{\strut R_2}&-\displaystyle\frac{\strut 1}{\strut R_1}-\frac{\strut 1}{\strut R_2}V_2+\frac{\strut 1}{\strut R_1}V_1\\
        0&\displaystyle\frac{\strut 1}{\strut R_2}+\frac{\strut 1}{\strut R_3}+\frac{\strut 1}{\strut R_4}&\displaystyle\frac{\strut 1}{\strut R_2}V_2\\
        1&-\displaystyle\frac{\strut1}{\strut R_3}&0
    \end{vmatrix}\\
    V_o=\displaystyle V_2R_2R_3\left[-\frac{1}{R_2^2}V_2+\frac{1}{R_1}\left(\frac{1}{R_2}+\frac{1}{R_3}+\frac{1}{R_4}\right)+\frac{1}{R_2}\left(\frac{1}{R_2}+\frac{1}{R_3}+\frac{1}{R_4}\right)\right]\\
    +V_1R_2R_3\left[-\frac{1}{R_1}\left(\frac{1}{R_2}+\frac{1}{R_3}+\frac{1}{R_4}\right)\right]
\end{gather*}
\begin{equation}
    V_o=\displaystyle\frac{(R_1+R_2)(R_3+R_4)+R_3R_4}{R_1R_3}V_2-\frac{R_2(R_3+R_3)+R_3R_4}{R_1R_4}V_1
\end{equation}

\subsection{Esercizio 1.2}

Determinare la potenza assorbita dal resistore $R_L$:
\begin{center}
    \begin{circuitikz}
        \draw (0,0) node[op amp](amp){}
        (-3,-2.75)node[ground]{}to[short](-3,-2.5)
        to[european voltage source=$1\,V$](-3,0.5)
        to[R=$3\mathrm{k\Omega}$,-d](-1,0.5)node[above left]{$A$}
        to[short](amp.-)
        (amp.+)to[short](-1.5,-0.5)
        to[short](-1.5,-2.5)
        to[short](-3,-2.5)
        (amp.down)to[short]++(0,-0.25)node[ground]{}
        (amp.out)to[short]++(5,0)
        to[R=${R_L={2}\mathrm{k\Omega}}$]++(0,-2.5)
        to[short]++(-2,0)coordinate(a)
        to[R=$15\mathrm{k\Omega}$]++(0,2.5)
        (a)to[short](-1.5,-2.5)
        (2,-2.5)to[R=$12\mathrm{k\Omega}$,-d](2,0)node[above right]{$B$}
        to[short](2,2)
        to[R=$6\mathrm{k\Omega}$](-1,2)
        to[short](-1,0.5);
    \end{circuitikz}
\end{center}
Per le leggi costitutive dell'amplificatore operazionale si ha $V_A=0$, per cui si può direttamente esprimere l'equazione risolutiva del metodo dei nodi, 
con già sostituita la corrente $i_x$ in uscita al posto di $V_A$:
\begin{gather*}
    \begin{bmatrix}
        0&-\displaystyle\frac{\strut1}{\strut6}\\1&\displaystyle\frac{\strut1}{\strut2}+\frac{\strut1}{\strut6}+\frac{\strut1}{\strut12}+\frac{\strut1}{\strut15}+
    \end{bmatrix}\begin{bmatrix}
        i_x\\V_B
    \end{bmatrix}=\begin{bmatrix}
        \displaystyle\frac{\strut1}{\strut3}\\0
    \end{bmatrix}
\end{gather*}
Si applica il metodo di Cramer:
\begin{gather*}
    V_B=6\begin{vmatrix}
        0&\displaystyle\frac{\strut1}{\strut3}\\1&0
    \end{vmatrix}=6\left(-\frac{1}{3}\right)=-2
\end{gather*}
La potenza si ottiene quindi come:
\begin{equation}
    P_{R_L}=\displaystyle\frac{V_B^2}{R_L}=\frac{4\mathrm{V}^2}{2\mathrm{k\Omega}}=2\mathrm{mW}
\end{equation}

\subsection{Esercizio 1.3}

Determinare il valore di $V_o$:
\begin{center}
    \begin{circuitikz}
        \draw (0,0)node[op amp](amp1){}
        (amp1.-)to[short,-d](-1.5,0.5)node[above left]{$A$}
        to[R=$1\mathrm{k\Omega}$](-3.5,0.5)
        (-3.5,-2.75)node[ground]{}to[short](-3.5,-2.5)
        to[european voltage source=$0.2\mathrm{V}$](-3.5,0.5)
        (amp1.+)to[short]++(-1,0)
        to[short]++(0,-2)
        (-1.5,0.5)to[short](-1.5,1.5)
        to[R=$10\mathrm{k\Omega}$](1,1.5)
        (amp1.out)to[short]++(0.25,0)
        to[short]++(0,1.5)
        (3,1)node[op amp, noinv input up](amp2){}
        (amp2.out)to[short,-o]++(0.5,0)node[above]{$V_o$}node[below]{$O$}
        (amp2.+)node[above]{$B$}to[short,d-](0.5,1.5)
        (-1.5,1.5)to[short](-1.5,2.5)
        to[R=$50\mathrm{k\Omega}$](4,2.5)
        to[short](4,-0.5)
        to[R=$1.33\mathrm{k\Omega}$](4,-2.5)
        to[short](-3.5,-2.5)
        (amp2.-)to[short,-*]++(0,-1)coordinate(a)node[left]{$P$}
        to[R=$1\mathrm{k\Omega}$]++(0,-2)
        (a)to[R=$4\mathrm{k\Omega}$](4,-0.5);
    \end{circuitikz}
\end{center}

La tensione in uscita al secondo operazionale $V_o$ è indipendente da ciò che entra nella sua entrata invertente. Passa una stessa corrente attraverso 
le due resistenze connesse al nodo $P$, poiché la corrente in entrata all'amplificatore è nulla, quindi non si biforca la corrente. Poiché la tensione di 
uscita $V_o$ è fissata, e misurata rispetto alla massa comune, si può determinare la tensione del nodo $P$ per la legge dei partitori di tensione, 
dato che le due resistenze sono virtualmente in serie:
\begin{gather*}
    V_P=1\mathrm{k\Omega}\displaystyle\frac{V_o}{5\mathrm{k\Omega}}=\frac{V_o}{5}
\end{gather*}
Per la legge costitutiva degli amplificatori operazionali, la tensione misurata in $P$ coincide alla tensione misurata sul nodo $B$, l'ingresso 
non invertente dell'amplificatore. 

Quindi è possibile risolvere questo circuito utilizzando solo tre nodi $A$, $B$ e $O$:
\begin{gather*}
    \begin{bmatrix}
        1+\displaystyle\frac{\strut1}{\strut 10}+\frac{\strut 1}{\strut 50}&-\displaystyle\frac{\strut 1}{\strut 10}&-\displaystyle\frac{\strut 1}{\strut 50}\\
        -\displaystyle\frac{\strut 1}{\strut 10}&\displaystyle\frac{\strut1}{\strut10}&0\\
        -\displaystyle\frac{\strut 1}{\strut 50}&0&G_o\\
    \end{bmatrix}\begin{bmatrix}
        V_A\\V_B\\V_o
    \end{bmatrix}=\begin{bmatrix}
        0.2\\-i{x_1}\\-i_{x_2}
    \end{bmatrix}
\end{gather*}
Si inseriscono i vincoli ottenuti precedentemente:
\begin{gather*}
    V_A=0\\
    V_B=\displaystyle\frac{V_o}{5}
\end{gather*}
L'equazione risolutiva diventa:
\begin{gather*}
    \begin{bmatrix}
        0&0&-\displaystyle\frac{\strut 1}{\strut 50}-\frac{\strut 1}{\strut 50}\\
        1&0&-\displaystyle\frac{\strut 1}{\strut 50}+0\\
        0&1&G_o\\
    \end{bmatrix}\begin{bmatrix}
        i_{x_1}\\i_{x_2}\\V_o
    \end{bmatrix}=\begin{bmatrix}
        0.2\\0\\0
    \end{bmatrix}
\end{gather*}
Si ricava la tensione in uscita tramite il metodo di Cramer:
\begin{gather}
    V_o=-25\begin{vmatrix}
        0&0&0.2\\
        1&0&0\\
        0&1&0\\
    \end{vmatrix}=-25\cdot0.2=-5\mathrm{V}
\end{gather}

\subsection{Esercizio 1.4}

Determinare l'espressione della corrente $i_x$:
\begin{center}
    \begin{circuitikz}
        \draw (0,0) node[op amp](amp){}
        (amp.out)to[short,-d](1.5,0)node[right]{$B$}
        to[short](1.5,1.5)
        to[R=$R_2$](-1.5,1.5)
        to[R=$R_1$](-3.5,1.5)
        to[short](-3.5,1)node[ground]{}
        (amp.-)to[short](-1.5,0.5)
        to[short,-d](-1.5,1.5)node[above]{$A$}
        (amp.+)to[short](-1.5,-0.5)
        to[short](-1.5,-1.5)
        to[R=$R_2$](1.5,-1.5)
        to[short](1.5,0)
        (-1.5,-1.5)node[above left]{$C$}to[R=$R_3$, i>_=$i_x$,d-](-1.5,-3.5)node[ground]{}
        (-3.5,-3.5)node[ground]{}to[european voltage source=$v_g$](-3.5,-1.5)
        to[R=$R_1$](-1.5,-1.5);
    \end{circuitikz}
\end{center}
Per la legge costitutiva degli amplificatori operazionali, si ha che $V_A=V_C$. 
Si risolve mediante il metodo dei nodi:
\begin{gather*}
    \begin{bmatrix}
        \displaystyle\frac{\strut1}{\strut R_1}+\displaystyle\frac{\strut1}{\strut R_2}&-\displaystyle\frac{\strut1}{\strut R_2}&0\\
        -\displaystyle\frac{\strut1}{\strut R_2}&\displaystyle\frac{\strut2}{\strut R_2}&-\displaystyle\frac{\strut1}{\strut R_2}\\
        0&-\displaystyle\frac{\strut1}{\strut R_2}&\displaystyle\frac{\strut1}{\strut R_1}+\displaystyle\frac{\strut1}{\strut R_2}+\displaystyle\frac{\strut1}{\strut R_3}
    \end{bmatrix}\begin{bmatrix}
        V_A\\V_B\\V_C=V_A
    \end{bmatrix}=\begin{bmatrix}
        0\\-i_{xo}\\\displaystyle\frac{\strut v_g}{\strut R_1}
    \end{bmatrix}\\
    \begin{bmatrix}
        \displaystyle\frac{\strut1}{\strut R_1}+\displaystyle\frac{\strut1}{\strut R_2}&-\displaystyle\frac{\strut1}{\strut R_2}&0\\
        -\displaystyle\frac{\strut1}{\strut R_2}-\displaystyle\frac{\strut1}{\strut R_2}&\displaystyle\frac{\strut2}{\strut R_2}&1\\
        \displaystyle\frac{\strut1}{\strut R_1}+\displaystyle\frac{\strut1}{\strut R_2}+\displaystyle\frac{\strut1}{\strut R_3}&-\displaystyle\frac{\strut1}{\strut R_2}&0
    \end{bmatrix}\begin{bmatrix}
        V_A\\V_B\\i_{xo}
    \end{bmatrix}=\begin{bmatrix}
        0\\0\\\displaystyle\frac{v_g}{R_1}
    \end{bmatrix}
\end{gather*}
Utilizzando il metodo di Cramer, si ottiene la seguente espressione per la tensione $V_A$:
\begin{gather*}
    V_A=V_C=\displaystyle\frac{1}{\displaystyle\frac{1}{R_2}\left(\frac{1}{R_1}+\frac{1}{R_2}\right)+\frac{1}{R_2}\left(-\frac{1}{R_1}-\frac{1}{R_2}-\frac{1}{R_3}\right)}
    \begin{vmatrix}
        0&-\displaystyle\frac{\strut1}{\strut R_2}&0\\
        0&\displaystyle\frac{\strut2}{\strut R_2}&1\\
        \displaystyle\frac{\strut v_g}{\strut R_1}&-\displaystyle\frac{\strut1}{\strut R_2}&0
    \end{vmatrix}\\
    \displaystyle\frac{\displaystyle\frac{1}{R_2}\left(-\frac{v_g}{R_1}\right)}{\displaystyle\frac{1}{R_2}\left(\frac{1}{R_1}+\frac{1}{R_2}\right)+\frac{1}{R_2}\left(-\frac{1}{R_1}-\frac{1}{R_2}-\frac{1}{R_3}\right)}\\
    V_A=\displaystyle\frac{v_gR_3}{R_1}
\end{gather*}
La corrente $i_x$ risulta quindi essere:
\begin{equation}
    i_x=\displaystyle\frac{V_A}{R_3}=\frac{v_g}{R_1}
\end{equation}

\subsection{Esercizio 1.5}

Determinare il valore di $v_o$:
\begin{center}
    \begin{circuitikz}
        \draw (0,0) node[op amp](amp){}
        (amp.+)to[short]++(-0.5,0)
        to[short]++(0,-1)
        (-5,-1.5)to[european voltage source=$12\mathrm{V}$](-5,0.5)
        to[R=$4\mathrm{k\Omega}$,-d](-3,0.5)
        (-5,-1.5)to[short,-o](1.5,-1.5)
        (-3,-1.5)to[european current source=$2\mathrm{mA}$,-d](-3,0.5)node[above right]{$A$}
        to[short](amp.-)
        (amp.out)to[short,-o](1.5,0)node[right]{$O$}
        (-3,0.5)to[short](-3,1.5)
        to[R=$3\mathrm{k\Omega}$](1,1.5)
        to[short](1,0);
        \draw[->](1.5,-1.25)--(1.5,-0.25)node[midway, right]{$v_o$};
    \end{circuitikz}
\end{center}
Poiché $V_A=0$, si indica come incognita la corrente in uscita dall'amplificatore $i_x$:
\begin{gather*}
    \begin{bmatrix}
        0&-\displaystyle\frac{\strut1}{\strut3}\\
        1&\displaystyle\frac{\strut1}{\strut3}
    \end{bmatrix}\begin{bmatrix}
        i_x\\V_o
    \end{bmatrix}=\begin{bmatrix}
        \displaystyle\frac{\strut12}{\strut4}+2\\0
    \end{bmatrix}
\end{gather*}
\begin{equation}
    V_o=3\begin{vmatrix}
        0&5\\1&0
    \end{vmatrix}=-15\mathrm{V}
\end{equation}

\subsection{Esercizio 1.6}

Determinare il valore di $v_o$:
\begin{center}
    \begin{circuitikz}
        \draw (0,0) node[op amp, noinv input up](a){}
        (a.out)to[short,-o](2,0)node[right]{$+$}node[below]{$v_o$}
        (1.5,0)to[R=$8\mathrm{k\Omega}$,-d](1.5,-2)node[right]{$A$}
        to[R=$4\mathrm{k\Omega}$](1.5,-4)
        to[short,-o](2,-4)node[right]{$-$}
        (a.-)to[short]++(0,-1.5)
        to[R=$6\mathrm{k\Omega}$](1.5,-2)
        (1.5,-4)to[short](-3,-4)
        to[european voltage source=$2.5\mathrm{V}$](-3,0.5)
        to[R=$6\mathrm{k\Omega}$](a.+);
    \end{circuitikz}
\end{center}
Poiché la corrente in entrata all'ingresso non invertente è nulla, per la legge costitutiva degli amplificatori, la tensione allo stesso 
ingresso coincide con la tensione erogata dal generatore a valle. 
Poiché anche la corrente in entrata all'ingresso invertente è nulla, la tensione che si trova al morsetto di ingresso, coincide alla tensione 
sul nodo $A$, poiché la resistenza che li collega non provoca una caduta di tensione, essendo attraversata da una corrente nulla. 
A questo punto per la legge del partitore di tensione inversa si ricava la tensione $v_o$, poiché la tensione $V_A$ è fissa:
\begin{gather*}
    V_A=\displaystyle\frac{4\mathrm{k\Omega}}{(8+4)\mathrm{k\Omega}}v_o\\
    v_o=3V_A=3\cdot2.5\mathrm{V}=7.5\mathrm{V}
\end{gather*}

\subsection{Esercizio 1.7}

Determinare il valore di corrente $i_o$: %%1:23:35
\begin{center}
    \begin{circuitikz}
        \draw (0,6) node[op amp, noinv input up](a1){A1}
        (0,0)node[op amp](a2){A2}
        (-3,-2.5)node[tlground]{}to[european voltage source=$5.8\mathrm{V}$](-3,-0.5)
        to[european voltage source=$2\mathrm{V}$](-3,6.5)
        to[R=$6\mathrm{k\Omega}$](a1.+)
        (-3,-0.5)to[R=$6\mathrm{k\Omega}$](a2.+)
        (a1.out)to[short](4,6)
        to[R=$4\mathrm{k\Omega}$,i>_=$i_o$](4,0)
        to[short](a2.out)
        (a1.-)to[short]++(0,-1.5)
        to[R=$6\mathrm{k\Omega}$,-d](1.5,4)node[right]{$A$}
        to[R=$8\mathrm{k\Omega}$,](1.5,6)
        (a2.-)to[short]++(0,1.5)
        to[R=$6\mathrm{k\Omega}$,-d](1.5,2)node[right]{$B$}
        to[R=$4\mathrm{k\Omega}$](1.5,4)
        (1.5,2)to[R=$8\mathrm{k\Omega}$,](1.5,0);
    \end{circuitikz}
\end{center}
Per la legge costitutiva degli amplificatori, il potenziale sulla porta in entrata di A1 è pari alla somma delle tensioni erogate tra i generatori:
\begin{gather*}
    v_{1,in}=7.8\mathrm{V}
\end{gather*}
Mentre il potenziale in entrata al secondo operazionale equivale alla tensione erogata dal generatore più a valle:
\begin{gather*}
    v_{2,in}=5.8\mathrm{V}
\end{gather*}
Poiché le correnti in entrata sono nulle, le resistenze di $6\mathrm{k\Omega}$ non provocano una caduta di tensione, per cui sui nodi $A$ e $B$ sono 
presenti rispettivamente i potenziali $v_{1,in}$ e $v_{2,in}$. Per cui la tensione sul resistore da $4\mathrm{k\Omega}$ si ottiene come la differenza 
tra i potenziali di $A$ e $B$:
\begin{gather*}
    V_{4\mathrm{k\Omega}}=V_A-V_B=v_{1,in}-v_{2,in}=2\mathrm{V}
\end{gather*}
La corrente che fluisce per la resistenza $4\mathrm{k\Omega}$ è la stessa che attraversa le due resistenza da $8\mathrm{k\Omega}$, 
per cui si può ricavare la tensione ai capi della tre resistenze in serie:
\begin{gather*}
    i_{4\mathrm{k\Omega}}=\displaystyle\frac{2\mathrm{V}}{4\mathrm{k\Omega}}=\frac{1}{2}\mathrm{mA}\\
    V_x=(8+8+4)\mathrm{k\Omega}\cdot\frac{1}{2}\mathrm{mA}=10\mathrm{V}
\end{gather*}
La corrente $i_o$ si ottiene quindi come:
\begin{equation}
    i_x=\displaystyle\frac{V_x}{4\mathrm{k\Omega}}=\frac{10}{4}\mathrm{mA}=2.5\mathrm{mA}
\end{equation}

\subsection{Esercizio 1.8}

Determinare il valore di $v_o$:
\begin{center}
    \begin{circuitikz}
        \draw (0,0) node[op amp, noinv input up](a){}
        (a.out)to[short,-o](2,0)
        (a.-)to[short]++(0,-1.5)
        to[short](1.5,-2)node[right]{$O$}
        to[R=$9\mathrm{k\Omega}$,d-](1.5,0)
        (1.5,-4)to[R=$1\mathrm{k\Omega}$](1.5,-2)
        (-1.5,-4)to[short,-o](2,-4)
        (1.5,-4)to[short](-6,-4)
        to[european voltage source=$2\mathrm{V}$](-6,0.5)
        to[R=$20\mathrm{k\Omega}$,-d](-2,0.5)node[above]{$A$}
        to[short](a.+)
        (-2,0.5)to[short](-2,-2)
        to[R=$40\mathrm{k\Omega}$](-2,-4)
        (-4,-4)to[european voltage source=$-5\mathrm{V}$](-4,-2)
        to[R=$40\mathrm{k\Omega}$](-2,-2);
        \draw[->](2,-3.75)--(2,-0.25)node[midway, right]{$v_o$};
    \end{circuitikz}
\end{center}

Questo circuito rappresenta un inseguitore di tensione, poiché l'amplificatore è connesso in modo tale che il potenziale inserito nel morsetto non invertente 
viene trasferito sul nodo $O$. Poiché la corrente in entrata è nulla, non è presente una retroazione, può essere risolto separatamente dall'amplificatore 
il circuito di sinistra per calcolare il potenziale del nodo $A$, in seguito si applica al nodo $O$ e tramite la legge dei partitori di tensione si ottiene 
la tensione in uscita dalla rete. 
Si calcola mediante il metodo dei nodi il potenziale $V_A$:
\begin{gather*}
    V_A=\displaystyle\frac{\displaystyle\frac{2}{20}-\frac{5}{40}}{\displaystyle\frac{1}{20}+\frac{1}{40}+\frac{1}{40}}\mathrm{V}=-\frac{1}{4}\mathrm{V}
\end{gather*}
La tensione in uscita si ottiene come:
\begin{gather*}
    V_o=\displaystyle\frac{(1+9)\mathrm{k\Omega}}{1\mathrm{k\Omega}}V_A=-2.5\mathrm{V}
\end{gather*}

\subsection{Esercizio 1.9}

Determinare la tensione in uscita $v_o$:
\begin{center}
    \begin{circuitikz}
        \draw (0,0) node[op amp](a){}
        (a.out)to[short,-o]++(1,0)node[right]{$v_o$}
        (a.-)to[short](-1.5,0.5)
        to[short,-d](-1.5,2.5)node[above]{$A$}
        to[R=$24\mathrm{k\Omega}$](1,2.5)
        to[short,-d](1,0)node[above right]{$O$}
        to[short](1,-0.5)
        to[R=$10\mathrm{k\Omega}$](1,-2.5)
        to[short](1,-2.75)node[tlground]{}
        (1,-2.5)to[short](-1.5,-2.5)
        to[european current source=$2\mathrm{mA}$,-d](-1.5,-0.5)node[above]{$B$}
        to[short](a.+)
        (-3.5,-0.5)to[R=$4\mathrm{k\Omega}$](-1.5,-0.5)
        (-3.5,-0.5)to[european voltage source=$4\mathrm{V}$](-3.5,-2.5)
        to[short](-5.5,-2.5)
        to[european voltage source=$3\mathrm{V}$](-5.5,2.5)
        to[R=$8\mathrm{k\Omega}$](-1.5,2.5)
        (-3.5,-2.5)to[short](-1.5,-2.5);
    \end{circuitikz}
\end{center}
Si calcola il potenziale sul nodo $B$, morsetto non invertente dell'operazionale, tramite la legge dei partitori di tensione:
\begin{gather*}
    V_B=\displaystyle\frac{\displaystyle-\frac{4}{4}+2}{\displaystyle\frac{1}{4}}\mathrm{V}=4\mathrm{V}
\end{gather*}
I potenziali ad entrambi i morsetti dell'operazionale sono uguali, per cui $V_A=V_B$, si inserisce questa equazione di vincolo direttamente nell'equazione 
risolutiva mediante il metodo dei nodi:
\begin{gather*}
    \begin{bmatrix}
        \displaystyle\frac{\strut1}{\strut8}+\frac{\strut1}{\strut24}&-\displaystyle\frac{\strut1}{\strut24}\\
        -\displaystyle\frac{\strut1}{\strut24}&\displaystyle\frac{\strut1}{\strut24}+\frac{\strut1}{\strut10}
    \end{bmatrix}\begin{bmatrix}
        4\\v_o
    \end{bmatrix}=\begin{bmatrix}
        \displaystyle\frac{\strut3}{\strut8}\\-i_x
    \end{bmatrix}\\
    \begin{bmatrix}
        0&-\displaystyle\frac{\strut1}{\strut24}\\
        1&G_o
    \end{bmatrix}\begin{bmatrix}
        i_x\\v_o
    \end{bmatrix}=\begin{bmatrix}
        \displaystyle\frac{\strut3}{\strut8}-\frac{\strut4}{\strut6}\\\displaystyle\frac{\strut1}{\strut6}
    \end{bmatrix}
\end{gather*}
Si calcola la tensione in uscita tramite il metodo di Cramer:
\begin{equation}
    v_o=24\begin{vmatrix}
        0&\displaystyle\frac{\strut9-16}{25}\\
        1&\displaystyle\frac{\strut1}{\strut6}
    \end{vmatrix}=7\mathrm{V}
\end{equation}

\clearpage

\end{document}